\section{Exact methods} \label{sec:exact}


In this section, we formulate an ILP model for Problem \ref{prob:min}, and discuss possible solution strategies. 

\subsection{Broadcast time model}
Given an integer $t\geq\tau(G,S)$, define the variables ($(u,v)\in \overrightarrow{E}$, $k=1,\ldots,t$) 
$$ x_{uv}^k=
\begin{cases} 
1, \text{ if } v\in V_k\setminus V_{k-1} \text{ and } \pi(v)=u,\\ 
0, \text{ otherwise},
\end{cases}
z_{k}=\begin{cases}
1, \text{ if } k\leq t,\\
0, \text{ otherwise}.
\end{cases}
$$
\noindent
Variable $x_{uv}^k$ thus represents the decision whether or not the signal is to be transmitted from node $u$ to node $v$ in period $k$,
while $z_k$ indicates whether transmissions take place as late as in period $k$.

An upper bound $t$ on the broadcast time $\tau(G,S)$ is easily available.
Because $G$ is connected, the cut between any set $V_i$ of informed nodes and its complement is non-empty,
and therefore at least one more node can be informed in each period.
It follows that $\tau(G,S)\leq n-\sigma$.
The bound is tight in the worst case instance where $S=\{v_1\}$, and $G$ is a path with $v_1$ as one of its end nodes.
Problem \ref{prob:min} is then formulated as follows: 
\begin{subequations}\label{mod:basic}
\begin{align}
\label{mod:basic:obj} \min \sum\limits_{k=1}^{t}z_k \\ 
\text{s. t.~~~} \label{mod:basic:singlein} \sum\limits_{k=1}^{t}\sum\limits_{v\in N(u)}x_{vu}^k  = 1 && u\in V \setminus S,\\
%\label{mod:basic:uniqueTout} \sum\limits_{v\in N(u)}x_{uv}^k & \leq 1  & u\in V,k=1,\dots,\bar{t},\\
\label{mod:basic:tIncreases} \sum\limits_{v\in N(u)}x_{uv}^k \leq\sum\limits_{\ell=1}^{k-1}\sum\limits_{w\in N(u)} x_{wu}^{\ell}  && u\in V\setminus S, k=2,\dots,t,\\
%\label{mod:basic:tIncreases} x_{uv}^k &\leq\sum\limits_{\ell=1}^{k-1}\sum\limits_{w\in N(u)\setminus\{v\}} x_{wu}^{\ell}  & \{u,v\}\in E, u\not\in S, k=2,\dots,\bar{t},\\
%\label{mod:basic:tcrel} \sum\limits_{k=1}^{\bar{t}}k\cdot x_{uv}^k & \leq t^* &  (u,v)\in A,\\
\label{mod:basic:tcrel} \sum\limits_{v\in N(u)}x_{uv}^k  \leq z_k &&  u\in V,k=1,\dots,t,\\
\label{mod:basic:timing} z_k  \leq z_{k-1} &&  k=2,\dots,t,\\
%\label{mod:basic:tcrel} \sum\limits_{t=1}^{n-1}t\sum\limits_{j\in N(i)}x_{ij}^k & \leq c &  i\in V,\\
\label{mod:basic:positiveCost}x_{uv}^1  = 0 && (u,v)\in \overrightarrow{E}, u \in V\setminus S,\\
\label{mod:basic:dec:0toSource} x_{uv}^k  = 0  && (u,v)\in\overrightarrow{E}, v\in S, k=1,\ldots,t\\
\label{mod:basic:dim}x \in \{0,1\}^{\overrightarrow{E}\times \{1,\dots,t\}},z\in\{0,1\}^{\{1,\ldots,t\}}.&&
\end{align}~
\end{subequations}
By \eqref{mod:basic:singlein}, every non-source node $u$ receives the signal from exactly one adjacent node $v$ in some time step $k$.
%The requirement that a non-source node has a neighbor $v\in V_k$ such that $\pi(v)=u$ only if there exists a node $w\in V_{k-1}$ such that $\pi(u)=w$ is modeled by \eqref{mod:basic:tIncreases}. 
The requirement that a non-source node $u$ informs a neighbor $v$ in the $k$-th time step only if $u$ is informed by some adjacent node $w$ in an earlier time step is modeled by \eqref{mod:basic:tIncreases}. 
%Constraints \eqref{mod:basic:uniqueTout} enforce that for each node $u\in V$ and each subset $V_k$, there is at most one adjacent node $v\in V_k$ with $\pi(v)=u$.
Constraints \eqref{mod:basic:tcrel} enforce that each node $u\in V$ forwards the signal to at most one adjacent node $v$ in each time step.
It also sets correct values to the $z$-variable that appear in the objective function.
%The requirement that only informed nodes can relay a signal is modeled by \eqref{mod:basic:tIncreases}. 
%The maximum time step at which any transmission takes place is captured by \eqref{mod:basic:tcrel}, and finally, \eqref{mod:basic:positiveCost} states that a node that is not a source never transmits in the first time step.
%The length of the sequence of subsets is captured by \eqref{mod:basic:tcrel}, and finally, \eqref{mod:basic:positiveCost} state that if $\pi(v)\not\in S$ for some $j\in V$, then $v\not\in V_1$.
The valid inequalities \ref{mod:basic:timing} reflect that $k\leq\tau(G,S)$ only if $k-1\leq\tau(G,S)$.
Lastly, constraints \eqref{mod:basic:positiveCost} and \eqref{mod:basic:dec:0toSource} state, respectively, that non-source nodes do not transmit in the first time step,
and that sources never receive the signal.

\subsubsection{Decision version}
\label{sec:decbasic}
The nature of MBT suggests another modelling approach derived from Model \eqref{mod:basic}. 
For a given positive integer $t$, we maximize the number of nodes $v$ that receive a signal from some neighbor $u$ within $t$ time steps.
Hence, $\tau(G,S)$ is the smallest value of $t$ for which the maximum attains the value $n-\sigma$.
If the optimal objective function value is $n-\sigma$, the corresponding values of the decision variable induce a broadcast forest with broadcast time $t$.
Otherwise, if the optimal objective function value is smaller than $n-\sigma$, $t<\tau(G,S)$, and Model \eqref{mod:basic:dec} is solved again with $t$ replaced by $t+1$.
The computational efficiency of a search where $t$ is increased from some lower bound on $\tau(G,S)$ is largely dependent of the tightness of available upper and lower bounds on $\tau(G,S)$.
If an upper bound $\bar{t}\geq \tau(G,S)$ is known, and it is revealed that for $t=\bar{t}-1$ fewer than $n-\sigma$ can receive the signal in time $t$, it is concluded that $\tau(G,S)=\bar{t}$.

The decision version of a model for \textsc{Minimum Broadcast Time} takes the form
\begin{subequations}\label{mod:basic:dec}
\begin{align}
\label{mod:basic:dec:obj} \max \sum\limits_{v \in V\setminus S}\sum\limits_{u \in N(v)} \sum\limits_{k=1}^{t}x_{uv}^k \\ 
\text{s. t.~~~} \label{mod:basic:dec:atMost1in} \sum\limits_{k=1}^{t}\sum\limits_{v\in N(u)}x_{vu}^k  \leq 1 && u\in V \setminus S,\\
\eqref{mod:basic:tIncreases},  \eqref{mod:basic:positiveCost}, \eqref{mod:basic:dec:0toSource}, \\
\label{mod:basic:dec:dim}x \in \{0,1\}^{\overrightarrow{E}\times \{1,\dots,t\}}.&&
\end{align}~
\end{subequations}

In the transition from the optimization model \eqref{mod:basic}, constraint \eqref{mod:basic:singlein} is replaced by \eqref{mod:basic:dec:atMost1in}.
The former is an inequality in the decision version, because not all nodes are necessarily reached within the given time limit.

