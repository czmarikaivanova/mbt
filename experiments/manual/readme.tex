\documentclass[12pt]{article}
\usepackage[small]{titlesec}
\title{MBT Experiments}
\begin{document}
\maketitle
This text contains a description of how to perform experiments with MBT ILP models.

\section{Structure of the folder \texttt{experiments/}}
\begin{itemize}
	\item\texttt{ampldata/ } - input files for the ILP models. The CPLEX solver is fed with these data files
	\item\texttt{javaprograms/ } - procedures written in java 
	\item\texttt{logs/ } - log files for elapsed time (\texttt{timelog.txt}) and objective function value (\texttt{objlog.txt})
	\item\texttt{manual/ } - this manual
	\item\texttt{models/ } - ILP models coded in AMPL
\end{itemize}

\section{Running the experiment}

The main script for running the experiments is \texttt{iterfiles.sh}.
It is run by typing \newline\newline
\texttt{bash ./iterfiles.sh [parameters]}\newline\newline
from the experiments directory (where iterfiles.sh is located).
There are three different modes in which \texttt{iterfiles.sh} can be executed.

\subsection{Input data from AMPL}

This option allows to process multiple AMPL files (with *.dat extension). \newline\newline
\texttt{bash ./iterfiles.sh ampl location\_of\_AMPL\_data\_files}\newline\newline
The first parameter specifies that the mode is AMPL and the second one points to the folder in which the data files are stored.
Typically it is the folder experiments/ampldata.

\subsection{Input data from 'raw' files}

A common way of how to store a graph in a file is shown in the following example:
\texttt{\newline
4 5\newline
0 1 1.00\newline
0 2 1.00\newline
0 3 1.00\newline
1 2 1.00\newline
2 3 1.00\newline
}
This example encodes that the graph has 4 nodes and 5 edges, each of unit weight. 
The edges are (0,1), (0,2), (0, 3),  (1,2) and  (2,3). 
In order to run experiments on files with raw format, use the command\newline\newline
\texttt{bash ./iterfiles.sh raw num\_of\_sources location\_of\_raw\_data\_files}\newline\newline

\subsection{Randomly generated instances}

The last option allows to run randomly generated instances with given parameters.
This option will probably not be used again, so it is only briefly described. 
More details will be added in case they are needed.
Type \newline\newline
\texttt{bash ./iterfiles.sh rand num\_of\_generated\_instances num\_of\_nodes num\_of\_sources density location\_of\_generated\_ampl\_files}\newline\newline
and the program will generate desired number of random graphs with given number of nodes, sources and edge density. 
Resulting AMPL files are then stored in the specified location.

\section{How it works}

The purpose of \texttt{iterfiles.sh} is to prepare AMPL data files and generate AMPL runfile that defines experimental settings and calls the solver which executes the computation.
Once the AMPL data files are prepared, the function \texttt{makeAMPLrun} is called. 
This function generates the AMPL run file with an AMPL data file as an input.

The file \texttt{iterfiles.sh} is very ad-hoc and contains several hard coded parameters (location of the cplex solver, log files, etc.), which should be adjusted according to the machine on which it is executed.
Which AMPL model is used is specified inside the body of the function \texttt{makeAMPLrun}.	
Elapsed time and objective function value of individual methods are stored in files \texttt{timelog.txt} and \texttt{objlog.txt}.



\end{document}
