\documentclass[12pt]{article}
\usepackage[small]{titlesec}
\title{MBT Experiments}
\begin{document}
\maketitle
This text contains a description of how to perform experiments with MBT ILP models.

\section{Structure of the directory \texttt{experiments/}}
\begin{itemize}
	\item\texttt{data/ } - instance files to be solved
	\item\texttt{logs/ } - log files for elapsed time (\texttt{timelog.txt}) and objective function value (\texttt{objlog.txt})
	\item\texttt{manual/ } - this manual
	\item\texttt{models/ } - ILP models coded in AMPL
	\item\texttt{old/ } - old version of the experiment scripts (no need to bother about them)
\end{itemize}

\section{Running the experiment}

The main script for running the experiments is \texttt{mbtrun.sh}.
The script is executed by typing \newline\newline
\texttt{./mbtrun.sh [path to directory with input files] <time limit> }\newline\newline
or\newline\newline 
\texttt{./mbtrun.sh [path to one input file] <time limit> }\newline\newline
from the directory \texttt{experiments/} (where \texttt{mbtrun.sh} is located).
The first parameter specifying either a directory containing instance files, or a single instance file is required.
If the optional parameter \texttt{<time limit>} in seconds is not specified, the default value (3600s) applies.

\section{Input data}

The input data files are not expected to follow any specific naming convention.
The instances is encoded as \newline\newline
\texttt{
cardV cardS cardE\newline
v1 v2\newline
v1 v3\newline
v2 v4\newline
...\newline
lb ub\newline\newline
}
where \texttt{cardV} is the number of nodes, \texttt{cardS} is the number of sources (the sources have indices $0\dots\texttt{cardS}-1$, 
and \texttt{cardE} is the number of edges (potential communication links).
Finally \texttt{lb} and \texttt{ub} denote lower and upper bound on the optimal solution, respectively.
There must not be any other characters.
An example of an input file representing a path on 9 nodes with one source "0" look as follows:\newline\newline
\texttt{
9	8	1\newline
0	1\newline
1	2\newline
2	3\newline
3	4\newline
4	5\newline
5	6\newline
6	7\newline
7	8\newline
4	12\newline\newline
}

\section{How it works}

The script iterates over all filenames in the input directory and reads the data from them.
We study two models, let's call them \emph{maxInformed} and \emph{minTime}.
Both models are written in \texttt{models/MBT-combined.mod}.
It contains two objective functions, and commands \texttt{drop [constraint\_name]} and \texttt{restore [constraint\_name]} specify which constraints are used for which model. 
Once the data are loaded, the script runs LP relaxation of the model \emph{maxInformed}, then the same model without the relaxation, and finally the model \emph{minTime}.

When running the model \emph{maxInformed}, the script iterates over increasing deadline. 
The iterative process terminates once either the objective value is \texttt{cardV-cardS}, or elapsed time exceeds a given time limit, or the number of iteration reaches an upper bound.

The file \texttt{mbtrun.sh} is very ad-hoc and contains several hard coded parameters (location of the cplex solver, log files, etc.), 
which should be adjusted according to the machine on which it is executed.

\section{Logging}

Elapsed time in seconds and objective function value of individual methods are stored in files \texttt{logs/timelog.txt} and \texttt{logs/objlog.txt}.
Each line starts with a filename of the processed instance followed by the respective values in the order mentioned above.

The reported elapsed time for model \emph{maxInformed} is a sum of elapsed times of all calls of the \texttt{solve} command during the iterative procedure.

\end{document}
