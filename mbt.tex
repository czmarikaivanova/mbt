\RequirePackage{fix-cm}
\documentclass[preprint,12pt, review]{elsarticle}
\usepackage[T1]{fontenc}

\usepackage{graphicx}
\usepackage{tkz-graph}
\usepackage{amsmath, amssymb}
\usepackage{amsthm}
\usepackage{multirow}
\usepackage{graphicx}
\usepackage{tikz}
\usepackage{amsfonts}
\usepackage{subcaption}
\usepackage{hyperref}

\usepackage{color}	% Different font colors
\usepackage{enumerate}
\usepackage{enumitem}
\usepackage{mathtools}
\usepackage{color}
\usepackage[linesnumbered]{algorithm2e}
\SetKwRepeat{Do}{do}{while}

\captionsetup{compatibility=false}

\newtheorem{theorem}{\textbf{Theorem}}
\newtheorem{observation}[theorem]{\textbf{Observation}}
\newtheorem{definition}{\textbf{Definition}}
\newtheorem{problem}{\textbf{Problem}}
\newtheorem{proposition}[theorem]{\textbf{Proposition}}
\newtheorem{remark}[theorem]{\textbf{Remark}}
\newtheorem{lemma}[theorem]{\textbf{Lemma}}

\tikzset{
every edge/.style={fill=none} 
every node/.style={shape=circle,fill=gray!40,draw }}

\journal{Discrete Applied Mathematics}
%
\begin{document}
\begin{frontmatter}

\title{Computing the broadcast time of a graph}

\author{Marika Ivanova\corref{cor1}}
\ead{Marika.Ivanova@uib.no}
\cortext[cor1]{Corresponding author}

\author{Dag Haugland\corref{cor2}}
\ead{Dag.Haugland@uib.no}

\address{Department of Informatics, University of Bergen, Norway}


\date{Received: date / Accepted: date}

\begin{abstract}
Given a graph and a subset of its nodes, referred to as source nodes, the minimum broadcast problem asks for the minimum number of steps in which a signal can be transmitted from the sources to all other nodes in the graph.
In each step, the sources and the nodes that already have received the signal can forward it to at most one of their neighbor nodes.
The problem has previously been proved to be NP-hard. 
In the current work, we develop a compact integer programming model for the problem.
We also devise procedures for computing lower bounds on the minimum number of steps required, along with methods for constructing near-optimal solutions.
Computational experiments demonstrate that in a wide range of instances with sufficiently dense graphs, the lower and upper bounds under study collapse.
In instances where this is not the case, the integer programming model proves strong capabilities in closing the remaining gap.

\begin{keyword}
Graph \sep Communication \sep Integer Programming \sep Bounds
\end{keyword}
\end{abstract}
\end{frontmatter}

\section{Introduction}
\label{intro}
The minimum broadcast time (MBT) problem consists of a set of communication nodes with a subset of source nodes. 
The task is to disseminate a signal to every node in a shortest possible time (broadcast time), while abiding by communication rules.
An \emph{informed} node is a node that has received the signal.
Otherwise, a node is \emph{uninformed}.
At the beginning, the set of informed nodes is exactly the set of sources.
An informed node can send the signal to an uninformed node if the two nodes are located within a communication vicinity of each other.

The time is divided into a finite number of time steps.
Every informed node can, at each time step, forward the signal to at most one uninformed neighbor node.
Therefore, the number of informed nodes can at most be doubled from one step to the next.
%\subsection{Motivation and Related Work}
%This communication protocol differs from various wireless communication models where a signal can be relayed to all nodes within a visibility range of a sender.
This communication protocol appears in various practical application such as communication among computer processors or telephone networks.
%The applications are however not confined to wired networks.
Situations where the signals have to cover large distances typically assume sending the signal to one neighbor at a time.
This is common in satellite communication.

\subsection{Literature overview}

Deciding whether an instance of MBT has a solution with delay at most $t$ has been shown to be NP-complete \cite{slater81}. 
For bipartite planar graphs with maximum degree 3, NP-completeness persists even if $t=2$ or if there is only one source \cite{jansen95}.
When $t=2$, the problem also remains NP-complete for cubic planar graphs \cite{middendorf93}, grid graphs with maximum degree 3,
complete grid graphs, chordal graphs, and for split graphs \cite{jansen95}. 
The single-source variant of the decision version of MBT is NP-complete for grid graphs with maximum degree 4, and for chordal graphs \cite{jansen95}.
The problem is known to be polynomial in trees \cite{slater81}.
Whether the problem is polynomial or NP-complete for split graphs with a single source was stated as an open questions in \cite{jansen95}, and has to the best of our knowledge not been answered yet.

A number of inexact methods, for both general and special graph classes, have been proposed in the literature during the last three decades.
One of the first works of this category \cite{scheuermann84} 
introduces an exact dynamic programming algorithm based on generating all maximum matchings in an induced bipartite graph.
Additional contributions of \cite{scheuermann84} are heuristic approaches for near optimal broadcasting.
From more recent works we mention \cite{hasson04}, which describes a meta heuristic algorithm for MBT, and provides a comparison with other existing methods.
The communication model is considered in an existing satellite navigation system in \cite{chu17}, where a greedy inexact methods is proposed together with a non-linear mathematical model.
Examples of additional efficient heuristics can be found e.g. in \cite{harutyunyan06,harutyunyan14,wang10,jimborean13}.

Approximation algorithms for MBT are studied in \cite{kortsarz95}. 
The authors argue that methods presented in \cite{scheuermann84} provide no guarantee on the performance, and show that the $n$-node wheel is an example of an unfavourable instance.
They introduce an $\mathcal{O}(\sqrt{n})$-additive approximation algorithm for broadcasting in general graphs.
They further provide approximation algorithms for several graph classes with small separators with approximation ratio proportional to the separator size times $\log n$.
Another algorithm with $\mathcal{O}\left(\frac{\log n}{\log \log n}\right)$-approximation\footnote{All logarithms in this paper are of base 2} ratio is given in~\cite{elkin03}.
Most of the works cited above consider a single source.

A related problem extensively studied in the literature is the minimum broadcast graph problem \cite{grigni91,mcgarvey16}. 
A broadcast graph supports a broadcast from any node to all other nodes in optimal time $\lceil\log n\rceil$.
For a given integer $n$, the problem is to find a broadcast graph of $n$ nodes such that e.g.\ the number of edges, or the maximum node degree, in the graph is minimized.
The authors of \cite{mcgarvey16} study ILP models for $c$-broadcast graphs, a generalization of the problem that allows transmission of the signal to at most $c$ neighbours in a single time step.



\section{Network Model and Definitions} \label{sec:def}

The communication network is represented by a connected graph $G=(V,E)$ and a subset $S\subseteq V$ referred to as the set of sources.
We denote the number of nodes and the number of sources by $n=|V|$ and $\sigma=|S|$, respectively. 

\begin{definition} \label{def:broadcasttime}
The \emph{broadcast time} $\tau(G,S)$ of a node set $S\subseteq V$ in $G$ is defined as the smallest integer $t\geq 0$ for which there exist
a sequence $V_0\subseteq\dots\subseteq V_t$ of node sets and a function $\pi:V\setminus S\to V$, such that:
\begin{enumerate}
  \item $V_0=S$ and $V_t=V$, \label{def:boundary}
  \item for all $v\in V\setminus S$, $\{v,\pi(v)\}\in E$, \label{def:edge}
  \item for all $i=1,\ldots,t$ and all $v\in V_i$, $\pi(v)\in V_{i-1}$, and \label{def:parent}
  \item for all $u,v\in V_i\setminus V_{i-1}$, $\pi(u)=\pi(v)$ only if $u=v$. \label{def:unique}
\end{enumerate}
\end{definition}

Referring to Section \ref{intro}, the node set $V_i$ is the set of nodes that are informed in time step $i$.
Initially, only the sources are informed ($V_0=S$), whereas all nodes are informed after $t$ time steps ($V_t=V$),
and the set of informed nodes is monotonously non-decreasing ($V_{i-1}\subseteq V_i$ for $i=1,\ldots,t$).
The parent function $\pi$ maps each node to the node from which it received the signal.
Conditions \ref{def:edge}--\ref{def:parent} of Definition \ref{def:broadcasttime} thus reflect that the sender is a neighbor node in $G$,
and that it is informed at an earlier time step than the recipient node.
Because each node can send to at most one neighbor node in each time step,
condition \ref{def:unique} states that $\pi$ maps the set of nodes becoming informed in step $i$ to distinct parent nodes.

The optimization problem in question is formulated as follows:
\begin{problem}[\textsc{Minimum Broadcast Time}]\label{prob:min}
Given $G=(V,E)$ and $S\subseteq V$, find $\tau(G,S)$.
\end{problem}

\begin{definition} \label{def:broadcastgraph}
For any $V_0,\ldots,V_t$ and $\pi$ satisfying the conditions of Definition \ref{def:broadcasttime},
the corresponding \emph{communication forest} is
the digraph $D=(V,A)$, where $A=\left\{\left(v,\pi(v)\right): v\in V\right\}$.
Each connected component of $D$ is a \emph{communication tree}.
\end{definition}

\noindent
It is easily verified that the communication trees are indeed arborescences, rooted at distinct sources, with arcs pointing away from the source.
Let $T(s)=\left(V(s),A(s)\right)$ denote the communication tree in $D$ rooted at source $s\in S$,
and let $T_i(s)$ be the subtree of $T(s)$ induced by $V(s)\cap V_i$.
Analogously, let $D_i$ be the directed subgraph of $D$ induced by node set $V_i$.
For the sake of notational simplicity, the dependence on $(V_0,\ldots,V_t,\pi)$ is suppressed when referring to the directed graphs introduced here.

The degree of node $v$ in graph $G$ is denoted by $\delta_G(v)$.
The set of neighbors of $v\in V$ in $G$ is denoted by $N_G(v)$.
% Whenever there is no risk of confusion, the subscript $G$ is omitted.
For a given subset $U\subseteq V$ of nodes, we define $N_G(U)=\bigcup_{v\in U}N_G(v)$.

Likewise, $N_D^+(v)=\left\{u\in V:(v,u)\in A\right\}$ and $N_D^-(v)=\left\{u\in V:(u,v)\in A\right\}$ denote the neighbor sets of node $v$ in $D$,
and $N_D(v)=N_D^+(v)\cup N_D^-(v)$.
Finally, we let $\delta^+_D(v)=\left|N_D^+(v)\right|$ and $\delta^-_D(v)=\left|N_D^-(v)\right|$ denote, respectively, the out-degree and the in-degree of $D$,
and we let $\delta_D(v)=\delta_D^+(v)+\delta_D^-(v)$


\section{Exact methods} \label{sec:exact}


In this section, we formulate an ILP model for Problem \ref{prob:min}, and discuss possible solution strategies. 

\subsection{Broadcast time model}
The studied model is a straightforward formulation of the problem.
Consider variables 
$$ x_{uv}^k=
\begin{cases} 
1, \text{ if } v\in V_k \text{ and } \pi(v)=u,\\ 
0, \text{ otherwise},
\end{cases}
z_{k}=\begin{cases}
1, \text{ if } k\leq\tau(G,S),\\
0, \text{ otherwise}.
\end{cases}
$$
The worst case scenario is when $G$ is a path $v_1,\dots,v_n$ with $S=\{v_1\}$. 
In such an instance, the necessary number of time steps is $n-1$, which gives a trivial upper bound $\bar{t}=n-s$ on the broadcast time.
Problem \ref{prob:min} is then formulated as follows: 
\begin{subequations}\label{mod:basic}
\begin{align}
\label{mod:basic:obj} \min \sum\limits_{k=1}^{\bar{t}}z_k \\ 
\text{s. t.~~~} \label{mod:basic:singlein} \sum\limits_{k=1}^{\bar{t}}\sum\limits_{v\in N(u)}x_{vu}^k & = 1 & u\in V \setminus S,\\
%\label{mod:basic:uniqueTout} \sum\limits_{v\in N(u)}x_{uv}^k & \leq 1  & u\in V,k=1,\dots,\bar{t},\\
\label{mod:basic:tIncreases} \sum\limits_{v\in N(u)}x_{uv}^k &\leq\sum\limits_{\ell=1}^{k-1}\sum\limits_{w\in N(u)} x_{wu}^{\ell}  & u\in V\setminus S, k=2,\dots,\bar{t},\\
%\label{mod:basic:tIncreases} x_{uv}^k &\leq\sum\limits_{\ell=1}^{k-1}\sum\limits_{w\in N(u)\setminus\{v\}} x_{wu}^{\ell}  & \{u,v\}\in E, u\not\in S, k=2,\dots,\bar{t},\\
%\label{mod:basic:tcrel} \sum\limits_{k=1}^{\bar{t}}k\cdot x_{uv}^k & \leq t^* &  (u,v)\in A,\\
\label{mod:basic:tcrel} \sum\limits_{v\in N(u)}x_{uv}^k & \leq z_k &  u\in V,k=1,\dots,\bar{t},\\
\label{mod:basic:timing} z_k & \leq z_{k-1} &  k=2,\dots,\bar{t},\\
%\label{mod:basic:tcrel} \sum\limits_{t=1}^{n-1}t\sum\limits_{j\in N(i)}x_{ij}^k & \leq c &  i\in V,\\
\label{mod:basic:positiveCost}x_{uv}^1 & = 0 & (u,v)\in A, u \in V\setminus S,\\
\label{mod:basic:dim}x \in \{0,1\}^{A\times \{1,\dots,\bar{t}\}},z&\in\{0,1\}^{\{1,\dots,\bar{t}\}}.&
\end{align}~
\end{subequations}
By \eqref{mod:basic:singlein}, every non-source node $u$ receives the signal from exactly one adjacent node $v$ in some time step $k$.
%The requirement that a non-source node has a neighbor $v\in V_k$ such that $\pi(v)=u$ only if there exists a node $w\in V_{k-1}$ such that $\pi(u)=w$ is modeled by \eqref{mod:basic:tIncreases}. 
The requirement that a non-source node $u$ informs a neighbor $v$ in the $k$-th time step only if $u$ is informed by some adjacent node $w$ in an earlier time step is modeled by \eqref{mod:basic:tIncreases}. 
%Constraints \eqref{mod:basic:uniqueTout} enforce that for each node $u\in V$ and each subset $V_k$, there is at most one adjacent node $v\in V_k$ with $\pi(v)=u$.
Constraints \eqref{mod:basic:tcrel} enforce that each node $u\in V$ forwards the signal to at most one adjacent node $v$ in each time step.
It also sets correct values to the $z$-variable that appear in the objective function.
%The requirement that only informed nodes can relay a signal is modeled by \eqref{mod:basic:tIncreases}. 
%The maximum time step at which any transmission takes place is captured by \eqref{mod:basic:tcrel}, and finally, \eqref{mod:basic:positiveCost} states that a node that is not a source never transmits in the first time step.
%The length of the sequence of subsets is captured by \eqref{mod:basic:tcrel}, and finally, \eqref{mod:basic:positiveCost} state that if $\pi(v)\not\in S$ for some $j\in V$, then $v\not\in V_1$.
The valid inequality \ref{mod:basic:timing} states that $k$ is no greater than broadcast time of the given node set only if the same holds for $k-1$.
Lastly, \eqref{mod:basic:positiveCost} state that non-source nodes do not transmit in the first time step.

\subsubsection{Decision version}
\label{sec:decbasic}
The nature of MBT suggests another modelling approach derived from Model \ref{mod:basic}. 
For a given maximum time period $t$, we maximize the number of nodes $v$ that receive a signal from some neighbor $u$ within $t$ time steps.
If the optimal value is $n-s$, it is clear that we found a spanning broadcast tree with broadcast time $t$.
In case the optimal value is less than $n-s$, the given $t$ is insufficient, and we have to try solving the problem with an increased time limit.
Clearly, an additional knowledge of upper and lower bound spares some computations. 
A lower bound is the initial value of $t$. 
Similarly, if an upper bound $\bar{t}$ is known and $t=\bar{t}-1$, the process can be terminated even when the objective function is less then $n-s$, because $\bar{t}$ is than the optimal value.

The decision model \ref{mod:basic:dec}  has the form
\begin{subequations}\label{mod:basic:dec}
\begin{align}
\label{mod:basic:dec:obj} \max \sum\limits_{v \in V\setminus S}\sum\limits_{u \in N(v)} \sum\limits_{k=1}^{t}x_{uv}^k \\ 
\text{s. t.~~~} \label{mod:basic:dec:atMost1in} \sum\limits_{k=1}^{t}\sum\limits_{v\in N(u)}x_{vu}^k & \leq 1 & u\in V \setminus S,\\
\label{mod:basic:dec:0toSource} \sum\limits_{k=1}^{t}\sum\limits_{v\in N(u)}x_{vu}^k & = 0  & u\in S,\\
\eqref{mod:basic:tIncreases},\notag \eqref{mod:basic:timing},  \eqref{mod:basic:positiveCost},\\
\label{mod:basic:dec:dim}x &\in \{0,1\}^{A\times \{1,\dots,t\}}.&
\end{align}~
\end{subequations}

Constraint \eqref{mod:basic:singlein} is replaced by \eqref{mod:basic:dec:atMost1in} and \eqref{mod:basic:dec:0toSource}. 
The former is an inequality, because not all nodes are necessarily reached within the given time limit.
The latter makes sure that sources are not reached by any signal. 
This was filtered out by optimality in \eqref{mod:basic}, but it must be forbidden explicitly in \eqref{mod:basic:dec} due to the changed objective function.
The actual target function is the minimum time required for spreading the signal to all nodes, and so we are looking for the smallest value of $t$.


\section{Lower bounds}
Strong lower bounds on the minimum objective function value are of vital importance to combinatorial optimization algorithms.
In this section, we study three types of lower bounds on the broadcast time $\tau(G,S)$.

\subsection{Analytical lower bounds} \label{sec:lbanalyt}
%An optimal solution is obtained by solving a sequence of decision problems with varying $t$. 
%It is therefore desirable to determine tight lower and upper bounds in order to arrive in the optimum after solving as few decision problems as possible.
Obvious bounds for a general graph instance are given by
\begin{observation}
For an instance $(G,S)$ of Problem \ref{prob:min},
\begin{equation}
\left\lceil\log\frac{n}{|S|}\right\rceil\leq \tau\{G,S\} \leq n-|S|.
\label{eq:loglb}
\end{equation}
\label{obs:loglb}
\end{observation}

In the following, we use $m$-step Fibonacci numbers \cite{noe05}, a generalization of the well known (2-step) Fibonacci numbers, defined by letting, 
$F^{(m)}_k=0$ for $k\leq 0$, $F^{(m)}_1=1$, and 
other terms according to the linear recurrence relation 
\begin{align*}
F^{(m)}_k &=\sum\limits_{i=1}^m F^{(m)}_{k-i}, &\text{ for } k\geq 2.
\end{align*}

% Possibly remove the subscript s in T_S
Assume that $G$ is a $d$-regular graph with a unique source $s$.
Any broadcast forest consists of a single tree $T_s$ rooted at $s$.
We investigate the number of leaves in $T_s$, and exploit this number to derive a lower bound on $\tau\{G,\{s\}\}$.

If the orientation of arcs in $T_s$ is disregarded, $|L(T^1_s)|=|L(T^2_s)|=2$.
For $i\geq 3$, $|L(T^i_s)|$ is no more than the number of nodes with degree below $d$ in $T^{i-1}_s$, 
because in a $d$-regular graph, only nodes with degree lower than $d$ can inform new uninformed nodes.
It can also be interpreted as the sum of the number of leaves in $T^{i-d+1}_s,\dots,T^{i-1}_s$, which leads to %the following formula
\begin{equation*}
\label{eq:leafrec}
|L(T^i_s)|=\sum\limits_{j=i-d+1}^{i-1} |L(T^j_s)|.
\end{equation*}  
This formula equals to the recursive definition of Fibonacci sequence of order $d-1$.
As each of the two base cases, $|L(T^1_s)|$ and $|L(T^2_s)|$, equals the double of the base cases of the Fibonacci sequence, the number of leaves in time step $i$ is calculated as
\begin{equation*}
\label{eq:fibleaf}
|L(T^i_s)|=2 F^{(d-1)}_i.
\end{equation*}  
The number of nodes in $T^i_s$ can be expressed as the sum of nodes newly informed in time steps $1,\dots,i$, that is, the sum of leaves in $T^1_s,\dots,T^i_s$. Thus,
\begin{equation}
\label{eq:fibcnt}
|V_{T^i_s}|=|V_i|=2\sum\limits_{j=1}^i F^{(d-1)}_j.
\end{equation}

\begin{proposition}
For a $d$-regular graph on $n$ nodes and $|S|$ sources, a lower bound on the delay is 
\begin{equation*}
\label{lem:lbreg1}
\underline{t}=\left\lceil\min\{k:2s\sum\limits_{j=1}^k F^{(d-1)}_j\geq n\}\right\rceil.
\end{equation*}
\label{prop:lbfib}
\end{proposition}
\begin{proof}
In order to inform $n$ nodes in the best possible scenario, the signal has to be relayed sufficiently many time steps so that the maximum possible number of informed nodes becomes $n$.
The maximum number of nodes informed within $i$ time steps is given by \eqref{eq:fibcnt}.
We therefore need to set the upper limit of the summation in \eqref{eq:fibcnt} so that the right-hand side exceeds $n$.
The reason why the result is divided by $|S|$ is that the best case scenario with several source nodes assumes that the signals initiated in individual sources are spread evenly.
\qed
\end{proof}

\subsection{Continuous relaxations of integer programming models} \label{sec:lblprel}

DH: \emph{To be written: Demonstrate why LP-bounds are expected to be weak (no stronger than the analytical bounds).}

\subsection{Combinatorial relaxations} \label{sec:lbcombrel}

Monotonous degree sequences of $G$ can be exploited more extensively than what was demonstrated in Section \ref{sec:lbanalyt}. 

Lower bounds on the broadcast time $\tau(G,S)$ are obtained by omitting one or more of the conditions imposed in Definition \ref{def:broadcasttime}.
For the purpose of strongest possible bounds, the relaxations thus constructed can be supplied with conditions that are redundant in the problem definition.

Recall from Section \ref{sec:def} that $D=(V,A)$ is the communication forest corresponding to $V_0,\ldots,V_t$, and $\pi$.
Conditions \ref{def:boundary}--\ref{def:unique} of Definition \ref{def:broadcasttime} imply that
\begin{enumerate}
\setcounter{enumi}{4}
  \item for all $v\in V$, $\delta_D^+(v)+\delta_D^-(v)\leq\delta_G(v)$. \label{def:degree}
\end{enumerate}

\noindent
A lower bound on $\tau(G,S)$ is then given by the solution to:
\begin{problem}[\textsc{Node Degree Relaxation}]\label{prob:degree}
Find the smallest integer $t\geq 0$ for which there exist
a sequence $V_0\subseteq\dots\subseteq V_t$ of node sets and a function $\pi:V\setminus S\to V$,
satisfying conditions \ref{def:boundary} and \ref{def:parent}--\ref{def:degree}.
\end{problem}

Except from the node degrees, explicit information about $G$ is not part of the input to Problem \ref{prob:degree},
and two nodes $u,v\in V\setminus S$, where $\delta_G(u)=\delta_G(v)$, are considered identical.

In order to develop an algorithm for the \textsc{Node Degree Relaxation}, we derive some optimality conditions of its single-source version.
Thus, let $|S|=\sigma$, $S=\left\{v_1,\dots,v_{\sigma}\right\}$ and $V\setminus S=\left\{v_{\sigma+1},\ldots,v_n\right\}$, where $\delta_G(v_{\sigma+1})\geq\delta_G(v_{\sigma+2})\geq\dots\geq\delta_G(v_n)$,
and denote $d_i=\delta_G(v_i)$ ($i=1,\ldots,n$).

Given a degree sequence 
\footnote{A degree sequence is commonly defined as a non-increasing sequence of degrees of nodes.
We however require, that first $\sigma$ positions correspond to degrees of sources, 
and the remaining $n-\sigma$ positions are degrees of the non-sources in a non-increasing order.} 
$Q=(d_1,\dots,d_n)$, we determine a LB by finding a smallest broadcast time of a graph that can be represented by $Q$.
Lemma \ref{lemma:degorder} characterizes such graphs.

\begin{lemma}
\label{lemma:degorder}
Maximum number of informed nodes within a given time limit $t\leq \bar{t} - \sigma$ in Problem~\ref{prob:degree} is achieved when nodes in $V\setminus S$ are informed 
in the order of their decreasing degree.
\end{lemma}
\begin{proof}

A node with degree $d_i$ informed in the $i$-th time step can inform at most $\min\{d_i-1,t-i\}$ nodes.
Consider integers  $k$ and $l$ such that $|S|< k < \ell\leq n$ and nodes $v_k$, $v_\ell$ with $\deg(v_k)=d_k$ and $\deg(v_\ell)=d_\ell$.
Then, $d_k-1\geq d_\ell-1$ and $t-k > t-\ell$.
If nodes are informed in the order of their decreasing degree, let $K_1$ and $L_1$ be the number of nodes that can be informed by $v_k$ and $v_l$, respectively:
$$
K_1=\min\{d_k-1,t-k\}, ~~~ L_1=\min\{d_\ell-1,t-\ell\}.
$$
If the order in which $v_k$ and $v_l$ are informed is switched, i.e. when $v_k$ and $v_\ell$ is informed in the $\ell$-th and $k$-th time step, respectively, 
we use $K_2$ and $L_2$ to denote the maximum number of nodes informed by $v_k$ and $v_\ell$:
$$
K_2=\min\{d_k-1,t-\ell\}, ~~~ L_2=\min\{d_\ell-1,t-k\}.
$$
We now investigate what values the difference 
\begin{equation}
\label{eq:degbounds}
(K_1+L_1)-(K_2+L_2)
\end{equation}
can attain.
Observe that $K_1\geq L_2$ and $K_2\geq L_1$. 
If $K_2>L_1$, there are two distinct cases:
Either $K_2=d_k-1=K_1$, and then also $L_2=d_l-1=L_1$.
Or $K_2=t-\ell\leq K_1$ implying $L_1 =d_\ell-1$, and since $d_\ell-1<t-\ell<t-k$, we have $L_2=d_\ell-1$ as well.
We conclude that \eqref{eq:degbounds} always takes a non-negative value.
%Therefore, if the nodes are informed in the order of their decreasing degrees,
%the number of informed nodes within the give time step is no worse than when the nodes are informed in any other order.
As long as it is possible to swap the order in which two nodes are informed so that a node with a higher degree becomes informed before a node with a lower degree,
thus whenever such $k$ and $l$ as above exist, swapping the order can not decrease the number of informed nodes within the given time limit.
\qed
\end{proof}
For the relaxed problem, informing nodes in the desired order is clearly possible.
If $\mathcal{Q}$ is the set of graphs represented by $Q$, a straightforward algorithm determining
$\min\{\tau(G,S): G\in \mathcal{Q}\}$,
and thus an optimal solution to Problem~\ref{prob:degree} can be developed.

Alg. \ref{alg:dreg} operates with the set $I(t)$ that contains nodes informed within the first $t$ time steps.
The function $a_i(t)$ determines the number of nodes informed by node $v_i$ within $t$ time steps, and finally,
the set $F(t)$ contains fertile nodes in time step $t$.
A node $v$ is called fertile in time step $t$ if $v$ informed less than $\delta_G(v)$ other nodes, and $v$ itself is also informed.

\begin{algorithm}
$\text{Let } I(0)=\{1,\dots,\sigma\}, a_1(0)=\dots = a_{n}(0)=0.$\\
\For{$t=1,2,\dots$} {
	$F(t)=\{i\in I(t-1):a_i(t-1)<d_i\},$\\
	$I(t)=\{1,\dots,|I(t-1)|+|F(t)|\},$\\
	$a_i(t)=a_i(t-1)+
	\begin{cases}
		1, i\in F(t)\cup \left(I(t)\setminus I(t-1)\right) \\
		0, \text{ otherwise. }\\
	\end{cases}$
}
$\text{Let } \tau(d_1,\dots,d_n,\sigma)=\min\{t=0,1,\dots:|I(t)|=n\}$
\caption{Lower bound exploiting distribution of degrees}
\label{alg:dreg}
\end{algorithm}


\begin{corollary}
Alg.~\ref{alg:dreg} finds a lower bound to Problem~\ref{prob:min}.
\label{cor:deg}
\end{corollary}

%A lower bound on $\tau(G,S)$ is then given by the solution to:
%\begin{problem}[\textsc{Node Degree Relaxation}]\label{prob:degree}
%Find the smallest integer $t\geq 0$ for which there exist
%a sequence $V_0\subseteq\dots\subseteq V_t$ of node sets and a function $\pi:V\setminus S\to V$,
%satisfying conditions \ref{def:boundary} and \ref{def:parent}--\ref{def:degree}.
%\end{problem}
%
%Except from the node degrees, explicit information about $G$ is not part of the input to Problem \ref{prob:degree},
%and two nodes $u,v\in V\setminus S$, where $\delta_G(u)=\delta_G(v)$, are considered identical.
%
%In order to develop an algorithm for the \textsc{Node Degree Relaxation}, we derive some optimality conditions of its single-source version.
%Thus, let $S=\left\{v_1\right\}$ and $V\setminus S=\left\{v_2,\ldots,v_n\right\}$, where $\delta_G(v_2)\geq\delta_G(v_3)\geq\dots\geq\delta_G(v_n)$,
%and denote $d_i=\delta_G(v_i)$ ($i=1,\ldots,n$).

%Consider an optimal communication tree $T$. That is, $T$ is a subtree of $G$ rooted at the unique source $v_1$ such that
%$\tau\left(G,\{v_1\}\right)=\tau\left(T,\{v_1\}\right)$.
%Let $\tau_i=\tau\left(T[i],\{v_i\}\right)$ denote the broadcast time of node $v_i$ in the subtree $T[i]$ of $T$ rooted at $v_i$.
%The optimal broadcast time $\tau\left(T,\{v_1\}\right)$ is hence no less than the sum of the time $\lceil\log\beta(v_i)\rceil$ it takes to inform $v_i$,
%and the subsequent time $\tau_i$ needed to inform all nodes in $T[i]$.
%Taking the largest of all such lower bounds, we get:
%\begin{equation}
%  \tau\left(T,\{v_1\}\right)=\max\left\{\lceil\log\beta(v_i)\rceil+\tau_i: i=2,\ldots,n\right\}. \label{eq:tau}
%\end{equation}
%
%\begin{lemma} \label{lem:degree}
%Problem \ref{prob:degree} has an optimal solution where, for all $i=2,\ldots,n-1$, $\delta_T^+(v_i)\geq\delta_T^+(v_{i+1})$ and $\tau_i\geq\tau_{i+1}$.
%\end{lemma}
%\begin{proof}
%Consider an optimal broadcast tree $T$, and let $j\in\{2,\ldots,n-1\}$ be the largest value of $i$ for which $\tau_i<\tau_{i+1}$.
%\emph{By utilizing (\ref{eq:tau}, go on to argue that a simple swap does not increase $\tau\left(\right)$}
%\emph{Slightly more tricky: Argue that moving subtrees can fix the first inequality. Swap again to restore the second}.
%\end{proof}


\section{Upper bounds}

A knowledge of an upper bound $\bar{t}$ affects the number of variables in all studied models. 
Particularly in the decision versions, the iterative approach can be terminated once the solution is found to be infeasible for broadcast time limit $\bar{t}-1$.
The algorithms presented in this section iteratively construct broadcast forest $T=(V_T,A_T)$, where in the last iteration $V_T=V$.
\subsection{Matching heuristic}

A fast straightforward algorithm determining an upper bound is based on repetitive construction of a matching between already connected nodes and their neighbors.
Initially, the constructed broadcast forest consists of isolated sources. 
In each iteration, the nodes already in the forest are sorted according to the number of neighbors not yet added to the forest.
For each node we then select a neighbor to be appended to the forest.
Let $V_T=\{v_1,\dots,v_{|V_T|}\}$.
The steps are formalized in Alg. \ref{alg:matchheur}.
%\begin{algorithm}[]
%\KwData{$G=(V,E), S\subseteq V$}
%\textbf{for }$s\in S\textbf{ do } V_s\leftarrow \{s\}, A_s\leftarrow\emptyset$\;
%$t\leftarrow 0$\;
%\While{$V_T\neq V$} {
%	Sort nodes in $V_T$ increasingly by the number of neighbors in $V\setminus V_T$\;
%	$W\leftarrow\emptyset$\;
%	\For{$u\in V_T$} {
%		Find $v\in V\setminus (V_T\cup W)$ such that $(u,v)\in A$\;
%		$W\leftarrow W\cup\{v\}$\;
%	}
%	$A_T\leftarrow A_T\cup \{\{u,w\}: u\in V_T, w\in N(u)\cap W\}$\;
%	$V_T\leftarrow V_T\cup W$\;
%	$t\leftarrow t+1$\;
%}
%\Return t\;
%%\Return $\lceil k/s \rceil$\;
% \caption{A greedy matching heuristic method for determining an upper bound}
%\label{alg:matchheur}
%\end{algorithm}

\begin{algorithm}[]
\KwData{$G=(V,E), S\subseteq V$}
$V_T\leftarrow S, A_T \leftarrow \emptyset$\;
$\bar{t}\leftarrow 0$\;
\While{$V_T\neq V$} {
	Sort nodes in $V_T$ so that $i<j\Leftrightarrow |N(v_i)\cap(V\setminus V_T)|<|N(v_j)\cap (V\setminus V_T)|$\;
	$W\leftarrow\emptyset$\;
	$D\leftarrow\emptyset$\;
	\For{$i=1,\dots,|V_T|$} {
		Find $v_j\in V\setminus (V_T\cup W)$ such that $(v_i,v_j)\in A$\;
		$W\leftarrow W\cup\{v_j\}$\;
		$D\leftarrow D\cup\{\left(v_i,v_j\right)\}$\;
	}
	$V_T\leftarrow V_T\cup W$\;
	$A_T\leftarrow A_T\cup D$\;
	$\bar{t}\leftarrow \bar{t}+1$\;
}
\Return $\bar{t}$\;
%\Return $\lceil k/s \rceil$\;
 \caption{A greedy matching heuristic method for determining an upper bound}
\label{alg:matchheur}
\end{algorithm}
The selection of a neighbor (line 7) can be done arbitrarily, or according to some strategy such as neighbor with maximum degree is selected first.
Note that for determining an upper bound, it is not necessary to store arcs sets $A_T$ and $D$, because we merely count the number of iterations.
In order to construct an actual solution with broadcast time equal to the upper bound, the arcs sets are needed.
Matchings created in each iteration are not necessarily maximum.

%\subsection{Restricted binomial tree method}
%
%The following inexact method is based on the idea of finding maximum cardinality matching in $G\left[V_T,V\setminus V_T\right]$, and extending $T$ by this matching.
%It means that in each iteration, the maximum possible number of nodes are informed.
%The maximum cardinality matching can be regarded as finding $|V_T|$ node-disjoint  binomial trees of order at most one with roots in $V_T$, maximizing the number of edges.
%By generalizing this notion, we iteratively look for $|V_T|$ node-disjoint (pruned) binomial trees of an arbitrary given order $k$ valued between 1 and $n-|S|$, but typically not more than 4.
%Even though this problem is NP-hard for $k\geq 2$ \cite{jansen95}, it is expected that the computational time is sound in most practical instances.
%After obtaining the set of binomial trees, first $p$ nodes in each tree are selected and added to the broadcast forest.
%The parameter $p\in \{1,\dots,2^k\}$ is also a part of the input.
%\begin{algorithm}[]
%\KwData{$G=(V,E), S\subseteq V, k\in \{1,\dots,n-|S|\}, p\in \{1,\dots,2^k\}$}
%$V_T\leftarrow S, A_T \leftarrow \emptyset$\;
%$\bar{t}\leftarrow 0$\;
%\While{$V_T\neq V$} {
%	$S\leftarrow V_T$\;
%	Find a set of pruned binomial trees $B=\{B_1,\dots,B_{|V_T|}\}$ of order at most $k$ with roots in $V_T$ by solving model \eqref{mod:genmatch}\;
%	$V_T\leftarrow V_T\cup \{v:v\in V_B:\beta(v)\leq p\}$\;
%	$A_T\leftarrow A_T\cup \{(u,v)\in A_B: \beta(u)\leq p,\beta(v)\leq p\}$\;
%	$\bar{t}\leftarrow \bar{t}+1$\;
%}
%\Return $\bar{t}$\;
%%\Return $\lceil k/s \rceil$\;
% \caption{A method for determining an upper bound based on iterative search for pruned binomial trees}
%\label{alg:match}
%\end{algorithm}
%
%Alg. \ref{alg:match} describes the process formally.
%Initially, the broadcast forest consists of isolated sources.
%The binomial trees are determined by solving the ILP model \eqref{mod:genmatch}. 
%Note that in every iteration, the model is solved with a different set $S$, as all nodes already included in the broadcast forest are roots of the binomial trees.
%The model considers the entire set $V$, but it is also possible to restrict this set to nodes with distance at most $k$ from some node in $V_T$.
%When this restriction is not imposed, nodes with larger distance are not a part of any binomial tree in the current iteration due to \eqref{mod:part:followArcsA} - \eqref{mod:part:followArcsB}.
%
%Like in Alg. \ref{alg:matchheur}, for calculating an upper bound, it is also not needed to store node set $A_T$ in Alg \ref{alg:match}, but it is essential when the actual broadcast tree is desirable.
%
\subsection{Restricted broadcast tree method}

The following idea is based on the observation that at every time step, the maximum number of nodes that can be informed equals the size of maximum matching between already informed nodes and the rest.
In the first iteration, the only informed nodes are the sources.
Once a maximum matching is found, the set of informed nodes is extended by the endpoits of the matching that were not yet informed.
This process is repeated until all nodes become informed.
The number of iteration necessary to inform all nodes is then the upper bound on the broadcast time.

A maximum matching can be found by an exact polynomial algorithm, or by solving an integer program.
Even though the second option is not a polynomial method, the solution time is negligible for the considered instance sizes.

Maximum matching can be found using integer programs presented earlier with maximum time step set to 1.
These models can be conveniently employed for an extension of this approach by increasing the maximum time step.
In our implementation, we used model \eqref{mod:basic:dec}.
A solution in each iteration gives a maximum number of newly informed nodes within the imposed time limit by finding a set of node disjoint broadcast trees rooted at nodes informed in previous iterations.
In this way we use the principle of rolling horizon method known from planning and schedulling.
For the next iteration, only some nodes are selected for extending the set of informed nodes, typically only the ones reachable in a single time step, thus a matching.
The steps are expressed by a pseudocode in Alg. \ref{alg:match}.
\begin{algorithm}[]
	\KwData{$G=(V,E), S\subseteq V, t_{\text{max}}\in \{1,\dots,n-|S|\}$}
$V_T\leftarrow S, A_T \leftarrow \emptyset$\;
$\bar{t}\leftarrow 0$\;
\While{$V_T\neq V$} {
	$S\leftarrow V_T$\;
	$x\leftarrow$ optimal solution to model \eqref{mod:basic:dec}\;
%	Find a set of restricted broadcast trees $\{T_1,\dots,T_{|V_T|}\}$ with broadcast time at most $t_{\text{max}}$ rooted in nodes of $V_T$ by solving model \eqref{mod:genmatch}\;
%	$V_T\leftarrow V_T\cup \{v:v\in V_B:\beta(v)\leq p\}$\;
	$V_T\leftarrow V_T\cup \{v\in V\setminus V_T:x_{uv}^1=1, u\in V_T\}$\;
	$A_T\leftarrow A_T\cup \{(u,v)\in V_T\times V\setminus V_T: x_{uv}^1=1\}$\;
	$\bar{t}\leftarrow \bar{t}+1$\;
}
\Return $\bar{t}$\;
%\Return $\lceil k/s \rceil$\;
 \caption{A method for determining an upper bound based on iterative search for  trees}
\label{alg:match}
\end{algorithm}

Like in Alg. \ref{alg:matchheur}, for calculating an upper bound, it is also not needed to store node and arcs sets $V_T$ and $A_T$ in Alg \ref{alg:match},
but it is essential when the actual broadcast tree is desirable.
It is also possible to use model \eqref{mod:genmatch} and obtain vector $y$.
In that case lines 6 and 7 in Alg. \ref{alg:match} would be replaced by $V_T\leftarrow V_T\cup\{v\in V\setminus V_T:y_{2u}^v=1,u\in V_T\}$ and 
$A_T\leftarrow A_T\cup \{(u,v)\in V_T\times V\setminus V_T:y_{2u}^v\}$, respectively.

\section{Experimental Results} \label{sec:exp}

The aim of our experiments is to evaluate computational abilities of B\&B applied to the studied model, and determine its usability on different instance types.
Also, we assess the strength of upper and lower bounding methods discussed above, as well as continuous relaxations of the models.

In some of the experimental settings we use randomly generated instances.
The generating procedure takes a number of nodes and a parameter $p$ as an input.
First, it generates a random tree with the given number of nodes.
It then iterates over all pairs of nodes that are not yet connected by an edge, and places an edge between them with the probability $p$.

Apart from randomly generated instances, we also evaluate data sets of various sizes and densities available online \cite{steinlib}.
These existing instances are used as benchmarks for algorithms for the minimum Steiner tree problem.

\subsection{Comparison of upper and lower bounds}

In the first set of experiments we study how does the strength of different methods depend on the parameter $p$.
We generate random single and double-source instances of sizes 125, 250, 500 and 1000 with increasing $p$.
Tabs. \ref{tab:obj-s1} and \ref{tab:obj-s2} summarize the results of randomly generated instances.
Each entry is calculated as an average from 100 instances for a given $n$, $p$ and a selected method.
Results obtained for existing instances are stated in Tab. \ref{tab:obj-exist} where 15 instances are used for obtaining average objective function value for each number of nodes.
Instead of the parameter $p$, the average number of edges in a given instance size is stated in the second column.

The column \emph{fib} contains Fibonacci bound.
The values are rarely higher than the trivial logarithmic bound.
Lower bound obtained from degree sequence in the column \emph{deg} is slightly better, 
but in majority of cases is far weaker than the LP bound, which is very tight in all considered experimental settings.
Upper bounds \emph{UB-t, $t=1,\dots 4$} are obtained using Alg. \ref{alg:match}, which takes $t$ as one of its input parameters.
In general, we observe that the higher $t$ the tighter upper bound is calculated.
There are however some instances where it does not hold, particularly for instances with larger $p$,  but the differences is very small, and can be explained as a coincidence.

We further observe that the span of bounds decreases with increasing $p$, within one instance size, and also increases for a constant $p$ with increasing instance size.
The experiments were not pursued for higher values of $p$, because it is very common that upper and lower bounds \emph{deg} and \emph{UB-4} coincide.
This behavior is easy to explain.
There are more possibilities how to relay a signal in dense graphs as compared to the sparser graphs. 
It is therefore likely that denser graphs have optimal broadcast time close to the lower bounds.
The decreasing span of bounds is also noticeable with increasing instance size.

\begin{table}[]
\centering
\begin{tabular}{rrrrrrrrrr}
n & p     & fib  & deg  & LP    & X-dec & UB-4  & UB-3  & UB-2  & UB-1  \\
\hline
\multirow{5}{*}{125} 
& 0.001 & 7.23 & 8.35 & 16.46 & 16.49 & 18.72 & 20.28 & 20.77 & 23.75 \\
& 0.002 & 7.21 & 8.20 & 13.97 & 13.98 & 15.80 & 17.18 & 17.21 & 19.50 \\
& 0.004 & 7.03 & 8.08 & 11.44 & 11.49 & 13.07 & 14.06 & 14.30 & 16.05 \\
& 0.008 & 7.00 & 7.86 & 9.09  & 9.22  & 10.71 & 11.25 & 11.85 & 12.90 \\
& 0.016 & 7.00 & 7.59 & 7.94  & 8.01  & 9.11  & 9.42  & 9.83  & 10.58 \\
%0.032 & 7.00 & 7.11 & 7.16  & 7.17  & 8.28  & 8.22  & 8.48  & 9.27  \\
%0.064 & 7.00 & 7.00 & 7.00  & 7.00  & 8.00  & 7.64  & 7.64  & 8.82 
\hline
\multirow{5}{*}{250} 
& 0.001 & 8.01 & 9.29 & 16.18 & 16.28 & 18.72 & 20.09 & 20.49 & 23.22 \\
& 0.002 & 8.00 & 9.17 & 12.87 & 12.93 & 15.23 & 16.19 & 16.84 & 18.78 \\
& 0.004 & 8.00 & 8.92 & 10.06 & 10.33 & 12.23 & 12.87 & 13.46 & 14.70 \\
& 0.008 & 8.00 & 8.83 & 8.96  & 9.03  & 10.31 & 10.73 & 11.34 & 12.18 \\
& 0.016 & 8.00 & 8.24 & 8.24  & 8.24  & 9.40  & 9.42  & 9.71  & 10.51 \\
%0.032 & 8.00 & 8.00 & 8.00  & 8.00  & 9.00  & 8.76  & 8.85  & 10.00
\hline
\multirow{5}{*}{500} 
& 0.001 & 9.01 & 10.20 & 14.35 & 14.54 & 17.06 & 18.47 & 18.92 & 20.96 \\
& 0.002 & 9.00 & 9.95  & 11.19 & 11.55 & 13.69 & 14.54 & 15.11 & 16.53 \\
& 0.004 & 9.00 & 9.86  & 10.09 & 10.12 & 11.81 & 12.12 & 12.69 & 13.57 \\
& 0.008 & 9.00 & 9.33  & 9.42  & 9.43  & 10.53 & 10.71 & 11.07 & 11.84 \\
& 0.016 & 9.00 & 9.00  & 9.00  & 9.00  & 10.00 & 9.94  & 10.00 & 11.06 \\
\hline
\multirow{5}{*}{1000} 
& 0.001 & 10.00 & 10.92 & 12.39 & 12.83 & 15.44 & 16.39 & 17.08 & 18.49 \\
& 0.002 & 10.00 & 10.77 & 11.06 & 11.07 & 12.99 & 13.38 & 14.00 & 14.98 \\
& 0.004 & 10.00 & 10.41 & 10.54 & 10.57 & 11.80 & 11.93 & 12.06 & 13.06 \\
& 0.008 & 10.00 & 10.01 & 10.01 & 10.01 & 11.02 & 11.02 & 11.02 & 12.04 \\
\end{tabular}
\caption{Objective function values yielded by different methods of randomly generated instances with $|S|=1$}
\label{tab:obj-s1}
\end{table}

\begin{table}[]
\centering
\begin{tabular}{rrrrrrrrrr}
n     &	p & fib  & deg  & LP    & X-dec & UB-4  & UB-3  & UB-2  & UB-1  \\
\hline
\multirow{5}{*}{125} 
& 0.001 & 6.26 & 7.10 & 13.33 & 13.35 & 15.20 & 16.70 & 16.88 & 19.23 \\
& 0.002 & 6.20 & 7.04 & 11.96 & 11.99 & 13.57 & 14.67 & 14.92 & 17.11 \\
& 0.004 & 6.04 & 7.01 & 9.75  & 9.77  & 11.14 & 12.05 & 12.36 & 13.73 \\
& 0.008 & 6.00 & 6.96 & 7.78  & 7.91  & 9.19  & 9.71  & 10.21 & 11.38 \\
& 0.016 & 6.00 & 6.78 & 6.90  & 6.97  & 8.10  & 8.27  & 8.62  & 9.40  \\
%& 0.032 & 6.00 & 6.09 & 6.16  & 6.17  & 7.14  & 7.13  & 7.44  & 8.26 
\hline
\multirow{5}{*}{250} 
& 0.001 & 7.03 & 8.14 & 14.59 & 14.63 & 16.92 & 18.07 & 18.40 & 20.67 \\
& 0.002 & 7.00 & 8.01 & 11.46 & 11.51 & 13.40 & 14.32 & 14.7  & 16.61 \\
& 0.004 & 7.00 & 7.98 & 8.99  & 9.16  & 10.94 & 11.57 & 12.11 & 13.29 \\
& 0.008 & 7.00 & 7.86 & 7.98  & 7.99  & 9.22  & 9.61  & 10.08 & 10.89 \\
& 0.016 & 7.00 & 7.26 & 7.27  & 7.27  & 8.33  & 8.35  & 8.61  & 9.44  \\
%0.032 & 7.00 & 7.00 & 7.00 & 7.00  & 7.00  & 8.00  & 7.69  & 7.94  & 9.02 
\hline
\multirow{5}{*}{500} 
& 0.001 & 8.00 & 9.04 & 12.82 &12.96  &15.47  & 16.47 & 17.05 & 18.86 \\
& 0.002 & 8.00 & 8.98 & 10.12 &10.29  &12.38  & 13.24 & 13.91 & 15.11 \\
& 0.004 & 8.00 & 8.87 & 8.98  &8.99   &10.61  & 11.03 & 11.47 & 12.41 \\
& 0.008 & 8.00 & 8.54 & 8.56  &8.56   &9.50   & 9.67  & 9.96  & 10.79 \\
& 0.016 & 8.00 & 8.09 & 8.09  &8.09   &8.98   & 8.94  & 8.98  & 10.05 \\
%0.032 & 8.00 & 8.00 & 8.00 & 8.00  &8.00   &8.05   & 8.06  & 8.05  & 10.00
\hline
\multirow{5}{*}{1000} 
& 0.001 & 9.00 & 10.00 & 11.23 & 11.44 & 14.02 & 14.79 & 15.56 & 16.95 \\
& 0.002 & 9.00 & 9.01  & 9.02  & 9.03  & 9.18  & 9.14  & 9.21  & 11.06 \\
& 0.004 & 9.00 & 9.51  & 9.59  & 9.61  & 10.83 & 10.97 & 11.10 & 12.06 \\
& 0.008 & 9.00 & 9.03  & 9.03  & 9.03  & 10.00 & 10.00 & 10.00 & 11.01 \\
& 0.016 & 9.00 & 9.00  & 9.00  & 9.00  & 9.12  & 9.07  & 9.15  & 11.00 \\
\end{tabular}
\caption{Objective function values yielded by different methods of randomly generated instances with $|S|=2$}
\label{tab:obj-s2}
\end{table}

\begin{table}[]
\centering
\begin{tabular}{rrrrrrrrrr}
n     &	avg $|E|$ & fib  & deg  & LP    & X-dec & UB-4  & UB-3  & UB-2  & UB-1  \\
\hline
  50 & 140  & 6.00 & 6.21 & 7.14 & 7.21 & 7.93 & 8.93  & 9.43  & 10.86 \\
 100 & 530  & 7.00 & 7.23 & 7.93 & 8.00 & 8.80 & 9.60  & 10.00 & 12.53 \\
 160 &3629  & 8.00 & 8.00 & 8.00 & 8.00 & 8.40 & 8.93  & 8.80  & 12.13 \\
 200 &3222  & 8.00 & 8.00 & 8.00 & 8.00 & 8.22 & 9.00  & 9.22  & 12.89 \\
 300 &7189  & 9.00 & 9.00 & 9.00 & 9.00 & 9.11 & 9.33  & 9.56  & 13.33 \\
 320 &2264  & 9.00 & 9.00 & 9.00 & 9.00 &10.33 &10.67  &10.73  & 11.87 \\
 400 &12612 & 9.00 & 9.00 & 9.00 & 9.00 & 9.22 & 9.78  & 9.78  & 13.89 \\
\end{tabular}
\caption{Objective function values yielded by different methods of existing instances available online with $|S|=1$}
\label{tab:obj-exist}
\end{table}


\subsection{Solution time}

Average solution time of instances from the previous section is reported in Tabs.~\ref{tab:soltime} and \ref{tab:soltime-exist}..
The columns LP and OPT contain solution times of solving LP relaxation and B\&B applied to the ILP model, respectively.
The column \emph{int.} contains the percentage of instances of which solving was interrupted after 1 hour, before the optimal solution was found.
The values in the last column \emph{col.} are the numbers of instances in which the upper and lower bounds collapsed. 
I.e., how many instances have the lower bound \emph{deg} value equal to the upper bound \emph{UB-4} value.
In these instances, the optimal solution is pursued, because its objective function value is already known.
Values in the second last column are therefore the proportion of interruptions in the total number of instances to which B\&B is actually applied 
(in which the bounds did not collapse).

All investigated instances of size up to 500 nodes are solved to optimality within the imposed 1 hour time limit.
Computation of B\&B on instances of 1000 nodes is often interrupted before the optimal solution is found, particularly those with large $p$.
Even thought the objective function value decreases with increasing $p$, and so the decision procedure needs to perform less iterations, overall solution time increases.
This growth is caused by larger number of $x$-variables in denser graphs.
It is also obvious and in accordance with the intuition that in denser graphs, the upper and lower bounds collapse more often.

\begin{table}[]
\begin{minipage}{.45\linewidth}
\begin{tabular}{rrrrrr}
\multicolumn{6}{c}{$|S|=1$} \\
$n$ & $p$ & LP  & OPT & int. & col. \\
\hline
\multirow{5}{*}{125} 
 & 0.001 & 0   &  0 & 0 & 0 \\ 
 & 0.002 & 0   &  0 & 0 & 0 \\
 & 0.004 & 0   &  0 & 0 & 0 \\
 & 0.008 & 0   &  0 & 0 & 0 \\
 & 0.016 & 0   &  1 & 0 & 4 \\
\hline                      
\multirow{5}{*}{250}        
 & 0.001 &  1  &  1 & 0 & 0 \\
 & 0.002 &  1  &  1 & 0 & 0 \\
 & 0.004 &  1  &  3 & 0 & 0 \\
 & 0.008 &  1  &  4 & 0 & 2 \\
 & 0.016 &  1  & 17 & 0 &20 \\
\hline                      
\multirow{5}{*}{500}        
 & 0.001 & 3   &  7 & 0 & 0 \\
 & 0.002 & 2   & 26 & 0 & 0 \\
 & 0.004 & 3   & 53 & 0 & 0 \\
 & 0.008 & 6   &214 & 0 &13 \\
 & 0.016 &11   &390 & 0 &10 \\
\hline                      
\multirow{5}{*}{1000}        
 & 0.001 & 12  &1366& 30& 0 \\
 & 0.002 & 20  & 814& 17& 0 \\
 & 0.004 & 56  &1637& 33&10 \\
 & 0.008 &101  &3399& 75& 6 \\
 & 0.016 &204  &3600&100&93 
\end{tabular}
\end{minipage}
\hspace{1.5cm}
\begin{minipage}{.45\linewidth}
\begin{tabular}{rrrrrr}
\multicolumn{6}{c}{$|S|=2$}  \\
$n$ & $p$ & LP  & OPT & int. & col. \\
\hline
\multirow{5}{*}{125} 
& 0.001 &  0  &  0 &  0  &  0\\
& 0.002 &  0  &  0 &  0  &  0\\
& 0.004 &  0  &  0 &  0  &  0\\ 
& 0.008 &  0  &  0 &  0  &  0\\
& 0.016 &  0  &  1 &  0  &  9\\ 
\hline
\multirow{5}{*}{250} 
& 0.001 &  0  &  1 & 0   & 0 \\
& 0.002 &  0  &  1 & 0   & 0 \\
& 0.004 &  0  &  2 & 0   & 0 \\
& 0.008 &  0  &  3 & 0   & 1 \\
& 0.016 &  1  & 13 & 0   &34 \\
\hline
\multirow{5}{*}{500} 
& 0.001 &  2  &  5 & 0   & 0 \\
& 0.002 &  2  & 22 & 0   & 0 \\
& 0.004 &  2  & 26 & 0   & 0 \\
& 0.008 &  4  &101 & 0   & 0 \\
& 0.016 &  6  &212 & 0   &11 \\
\hline
\multirow{5}{*}{1000} 
& 0.001 &  9  &866   & 17 & 0  \\
& 0.002 &  24 &378   & 0  & 0  \\
& 0.004 &  69 &1647  & 30 & 8  \\
& 0.008 &  107&3378  & 74 &30  \\
& 0.016 &  235&3600  &100 &88 
\end{tabular}
\end{minipage}
\caption{Solution time of LP relaxation and B\&B of randomly generated instances}
\label{tab:soltime}
\end{table}
			 

\begin{table}[]
	\centering
\begin{tabular}{rrrrr}
$n$ & avg $|E|$ & LP  & OPT &  col. \\
\hline
  50 & 140  & 0.02&0.14 & 26.67\\
 100 & 530  & 0.12&0.65 & 33.33\\
 160 &3629  & 0.12&1.75 & 73.33\\
 200 &3222  & 0.76&4.90 & 77.78\\
 300 &7189  & 7.43&15.69& 88.89\\
 320 &7189  & 2.66& 9.75& 26.67\\
 400 &12612 &12.83&74.08& 77.78
\end{tabular}
\caption{Solution time of LP relaxation and B\&B of existing instances available online for $|S|=1$}
\label{tab:soltime-exist}
\end{table}


\section{Concluding Remarks} \label{sec:conc}

This work focuses on the minimum broadcast time problem and presents several techniques for determining lower bounds, upper bounds as well as optimal solutions.
The main contribution consists in introducing two different integer programming models and their comparison.
We consider various instance types and sizes both from existing datasets and randomly generated.


% Non-BibTeX users please use
\begin{thebibliography}{}
%
% and use \bibitem to create references. Consult the Instructions
% for authors for reference list style.
%
\bibitem{chu17}
Chu, X., Chen, Y.,
Time division inter-satellite link topology generation problem: Modeling and solution,
International Journal of Satellite Communications and Networking, 194 -- 206, 36 (2017)

\bibitem{cormen90}
Cormen, T. H., Leiserson, C. E., Rivest, R. L,
Introduction to Algorithms, 
MIT Press, 401 -- 402, 1990. 

\bibitem{elkin03}
Elkin, M., Kortsarz, G.,
Sublogarithmic approximation for telephone multicast: path out of jungle,
Symposium on Discrete Algorithms, 76 -- 85 (2003)

\bibitem{farley81}
Farley, A. M., Proskurowski, A.,
Broadcasting in Trees with Multiple Originators,
SIAM Journal on Algebraic Discrete Methods, 381 -- 386, 2, 4 (1981)

\bibitem{grigni91}
Grigni, M., Peleg, D.,
Tight bounds on minimum broadcast networks
Networks, 207-222, 4 (1991)

\bibitem{hasson04} 
Hasson, Y., Sipper, M.,
A Novel Ant Algorithm for Solving the Minimum Broadcast Time Problem,
International Conference on Parallel Problem Solving from Nature, 775 -- 780 (2004)

\bibitem{harutyunyan06}
Harutyunyan, H. A., Shao, B.,
An efficient heuristic for broadcasting in networks,
Journal of Parallel and Distributed Computing, 68 -- 76, 66, 1 (2006)

\bibitem{harutyunyan14}
Harutyunyan, H. A., Jimborean, C.,
New Heuristic for Message Broadcasting in Network,
IEEE 28th International Conference on Advanced Information Networking and Application, 517 -- 524, (2014)

\bibitem{jansen95}
Jansen, K., M\"uller, H.,
The minimum broadcast time problem for several processor networks, 
Theoretical Computer Science, 69 -- 85, 147 (1995)

\bibitem{kortsarz95}
Kortsarz, G., Peleg, D.,
Approximation algorithms for minimum-time broadcast
SIAM Journal on Discrete Mathematics, 401 -- 427, 8, 3 (1995)

\bibitem{mcgarvey16}
McGarvey, R. G., Rieksts, B. Q., Ventura, J. A., Ahn, N.,
Binary linear programming models for robust broadcasting in communication networks,
Discrete Applied Mathematics, 173 -- 84, 204, (2016)

\bibitem{middendorf93}
Middendorf, M.,
Minimum broadcast time is NP-complete for 3-regular planar graphs and deadline 2,
Information Processing Letters, 281 -- 287, 46 (1993)

\bibitem{noe05}
Noe, T. D., Post, J. V., 
Primes in Fibonacci n-step and Lucas n-Step Sequences,
J. Integer Seq. 8, Article 05.4.4, 2005.

\bibitem{scheuermann84}
Scheuermann, P., Wu, G.,
Heuristic Algorithms for Broadcasting in Point-to-Point Computer Networks,
IEEE Transactions on Computers, 804 -- 811, 33, 9 (1984)

\bibitem{steinlib}
http://steinlib.zib.de/download.php

\bibitem{slater81}
Slater, P. J., Cockayne, E. J., Hedetniemi, S.T.,
Information dissemination in Trees,
SIAM Journal on Computing, 692 -- 701, 10, 4 (1981)

\bibitem{wang10}
Wang, W.,
Heuristics for Message Broadcasting in Arbitrary Networks,
Master thesis, Concordia University, Montr\'eal, Qu\'ebec, 
Retrieved from http://citeseerx.ist.psu.edu/viewdoc/download?doi=10.1.1.633.5827\&rep=rep1\&type=pdf (2010)

\bibitem{jimborean13}
Jimborean, C.,
New Heuristics for Message Broadcasting in Arbitrary Networks,
Master thesis, Concordia University, Montr\'eal, Qu\'ebec, 
Retrieved from https://spectrum.library.concordia.ca/977717/1/Jimborean\_MCompSc\_F2013.pdf (2013)


% Format for Journal Reference
%Author, Article title, Journal, Volume, page numbers (year)
% Format for books
%\bibitem{RefB}
%Author, Book title, page numbers. Publisher, place (year)
% etc
\end{thebibliography}

\end{document}
% end of file template.tex

