\section{Upper bounds} \label{sec:ub}

Access to an upper bound $\bar{t}\geq\tau(G,S)$ affects the number of variables in the models studied in Section \ref{sec:optbasic}. 
Particularly in the decision version \eqref{mod:basic:dec},
the iterative approach can be terminated once the solution is found to be infeasible for broadcast time limit $\bar{t}-1$.
Algorithms that output feasible, or even near-optimal solutions, are instrumental in the computation of upper bounds.
Further, such methods are required in sufficiently large instances, where exact approaches fail to terminate in practical time.

\subsection{Existing heuristic methods} \label{sec:heur}
% Following reviewer 1, we should describe briefly the methods in \cite{harutyunyan06, harutyunyan14, harutyunyan10}.
% One or more of these should also be studied in the experimental section.

Building on earlier works \cite{harutyunyan06, harutyunyan10},
Harutyunyan and Jimborean \cite{harutyunyan14} study a heuristic (considering $\sigma=1$) departing from a shortest-path tree of $G$.
A sequence of local improvements is performed in the bottom-up direction in the tree, starting by the leafs and terminating at the root node.
Rearrangements of the parent assignments are made in order to reduce the broadcast time needed in subtrees.
The heuristic has running time $\mathcal{O}\left(|E|\log{n}\right)$.
% Reference \cite{harutyunyan10} studies a heuristic based on breadth-first, and seems to output something like the shortest-path tree.
% The method in \cite{harutyunyan14} seems to be an improvement over the one in \cite{harutyunyan10}, and therefore a natural choice to benchmark against.
% Note that it is a fast method, with predictable, polynomial upper bounds $\mathcal{O}(|E|\log|V|)$ on the running time.
% As long as competing methods do not come with comparable bounds on the running time, they should output better solutions.

\subsection{A construction method} \label{sec:cons}

Consider an integer $t'\geq 0$, node sets $S=V_0\subseteq V_1\subseteq\cdots\subseteq V_{t'}\neq V$ and a function $\pi: V\setminus S\mapsto V$,
where $\left\{\pi(v),v\right\}\in E$ for all $v\in V_{t'}\setminus S$,
and conditions \ref{def:parent}--\ref{def:unique} of Definition \ref{def:broadcasttime} are satisfied for $t=t'$.
That is, $\left(V_0,\ldots,V_{t'},\pi\right)$ defines a broadcast forest corresponding to the instance $\left(G\left[V_{t'}\right],S\right)$,
but the forest does not cover $V$.
In particular, if $t'=0$, the broadcast forest is a null graph on $S$, while it is a matching from $S$ to $V_1\setminus S$ if $t'=1$.

This section addresses the problem of extending the partial solution $\left(V_0,\ldots,V_{t'},\pi\right)$ by another node set $V_{t'+1}$,
such that the conditions above also are met for $t=t'+1$.
With $t'=0$ as departure point, a sequence of extensions results in a broadcast forest corresponding to instance $(G,S)$.
Each extension identifies a matching from $V_{t'}$ to $V\setminus V_{t'}$, and all matched nodes in the latter set are included in $V_{t'+1}$.
A key issue is how to determine the matching.

Since the goal is to minimize the time (number of extensions) needed to cover $V$, a \emph{maximum cardinality} matching between
$V_{t'}$ and $V\setminus V_{t'}$ is a natural choice.
Lack of consideration of the matched nodes' capabilities to inform other nodes is however an unfavorable property.
Each iteration of Alg.\ \ref{alg:match} rather sees $k\geq 1$ time periods ahead, and maximizes the total number of nodes in $V\setminus V_{t'}$
that can be informed in periods $t'+1,\ldots,t'+k$.
Commitment is made for only one period, and the matched nodes are those that are informed in period $t'+1$ from some node in $V_{t'}$.
The maximization problem in question is exactly the one addressed by model \eqref{mod:basic:dec},
where $V_{t'}$ is considered as sources, $k$ the upper bound on the broadcast time, and the graph is $G$ with all edges within $V_{t'}$ removed.
Choosing $k=1$ corresponds to the maximum matching option, whereas large values of $k$ (e.g., $k=n-\sigma$) makes Alg.\ \ref{alg:match} an exact method. 

\begin{algorithm}[]
	\KwData{$G=(V,E), S\subseteq V, k\in \{1,\dots,n-\sigma\}$}
\For{$t'=0,1,\dots$} {
	\textbf{if} ~$S=V$ ~\textbf{return} $t'$ \\
	$x\leftarrow$ optimal solution to the instance of model \eqref{mod:basic:dec} with $t=k$ \\
	\For{$\{u,v\}\in E$ such that $u\in S$, $v\in V\setminus S$, and $x_{uv}^1=1$} {
		$S\leftarrow S\cup\{v\}$
	}
}
\caption{Construction of near-optimal solutions through sequences of matchings}
\label{alg:match}
\end{algorithm}

In many instances, there are multiple options for selecting an optimal solution $x$ (line 3).
Two different sequences of solutions generated during the course of Alg.~\ref{alg:match} may result in different broadcast times.
Let us take a hypercube $C_j$ as an example which has $\tau(C_j,S)=j$ for $\sigma=1$ regardless of selection of the source node.
Alg.~\ref{alg:match} with $k=1$ always finds $\tau(C_j,S)$ for $j=1\dots 3$. 
It also finds the optimum for $j\geq 4$ 
provided that the subgraph induced by the current $S$ at the beginning of each iteration $t'$ forms a hypercube $C_{t'}$.
If this tie-breaking rule is violated, the algorithm may fail to find the optimum as demonstrated in $C_4$ (TODO: image or description in the text?)

Intuitively, the broadcast time found by Alg.~\ref{alg:match} is likely to be closer to the optimum for larger $k$.
This intuition is also confirmed by a vast majority of instances in the experimental section.
Nevertheless, there are instances for which Alg.~\ref{alg:match} with smaller $k$ finds a better solution.
One such example is a tree $T$ on 17 nodes with one source $s$ and $\text{deg}(s)=2$.
The first neighbor $n_1$ of $s$ is a root of a binomial tree $B_2$ of degree two. 
One of the leafs of $B_2$ is an endpoint of a path of length 3.
The second neighbor $n_2$ of $s$ is adjacent to a root of binomial tree $B_3$.
Alg.~\ref{alg:match} with $k=3$ decides that $n_1\in V_1$, $n_2\in V_2$ and eventually finds $\tau{T,\left\{s\right\}}$.
With $k=5$, $n_1\in V_2$, $n_2\in V_1$, which results in a suboptimal solution.
(TODO: is such an example useful? Should there be an illustration?)


\begin{remark} \label{rem:time}
If $k=1$, then the running time of Alg.\ \ref{alg:match} is $\mathcal{O}\left(n^{\frac{3}{2}}|E|\right)$, because the number of iterations is no more than $n$,
and the maximum cardinality matching is found in $\mathcal{O}\left(\sqrt{n}|E|\right)$ time \cite{hopcroft73}.
For fixed $k=2$, the problem solved in each iteration is NP-hard \cite{jansen95}.
\end{remark}

\begin{remark} \label{rem:mvm}
If $k=1$, no attempt is made to favor the nodes with many uninformed neighbors.
To that end, maximum cardinality matching can be replaced by maximum \emph{vertex-weight} matching (MVM),
where each node $v\in V\setminus S$ is assigned the weight $1+\left|\left\{u\in V\setminus S: \{u,v\}\in E\right\}\right|$.
Other weights reflecting the capability of node $u$ to inform other nodes in later periods could also be considered.
The running time of Alg.\ \ref{alg:match} increases to $\mathcal{O}\left(n^2|E|\right)$, as MVM
is solved in $\mathcal{O}\left(n|E|\right)$ time \cite{dobrian19}.
An approximate MVM-solution within $\frac{2}{3}$ of optimality is found in $\mathcal{O}\left(|E|+n\log{n}\right)$ time \cite{dobrian19}.
\end{remark}
