\section{Lower bounds} \label{sec:lb}
Strong lower bounds on the minimum objective function value are of vital importance to combinatorial optimization algorithms.
In this section, we study three types of lower bounds on the broadcast time $\tau(G,S)$.

\subsection{Analytical lower bounds} \label{sec:lbanalyt}
Any solution $\left(V_0,\ldots,V_t,\pi\right)$ satisfying conditions \ref{def:boundary}--\ref{def:unique} of Definition \ref{def:broadcasttime},
also satisfies $\left|V_{k+1}\right|\leq 2\left|V_k\right|$ for all $k\geq 0$.
Combined with the observation made in Section \ref{sec:decbasic}, this yields the following bounds:
\begin{observation}
For all instances $(G,S)$ of Problem \ref{prob:min},
\begin{equation}
\left\lceil\log\frac{n}{\sigma}\right\rceil\leq \tau(G,S) \leq n-\sigma.
\label{eq:loglb}
\end{equation}
\label{obs:loglb}
\end{observation}

Consider the $m$-step Fibonacci numbers $\left\{f^{m}_k\right\}_{k=1,2,\ldots}$ \cite{noe05}, a generalization of the well-known (2-step) Fibonacci numbers, defined by
$f^{m}_k=0$ for $k\leq 0$, $f^{m}_1=1$, and 
other terms according to the linear recurrence relation 
\begin{align*}
f^{m}_k &=\sum\limits_{j=1}^m f^{m}_{k-j}, &\text{ for } k\geq 2.
\end{align*}

The generalized Fibonacci numbers are instrumental in the derivation of a lower bound on $\tau(G,S)$,
depending on the maximum node degree $d=\max\left\{\delta_G(v): v\in V\right\}$ in $G$.
The idea behind the bound is that the broadcast time can be no shorter than what is achieved if
the following ideal, but not necessarily feasible, criteria are met:
Every source transmits the signal to a neighbor node in each of the periods $1,\ldots,d$,
and every node $u\in V\setminus S$
transmits the signal to a neighbor node in each of the first $d-1$ periods following the period when $u$ gets informed.
An exception possibly occurs in the last period, as there may be fewer nodes left to be informed than there are nodes available to inform them.

\begin{proposition}
\begin{equation*}
\label{lem:lbreg1}
	\tau(G,S)\geq\min\left\{t:2\sigma\sum\limits_{j=1}^tf^{d-1}_j\geq n\right\}.
\end{equation*}
\label{prop:lbfib}
\end{proposition}
\begin{proof}

Consider a solution $\left(V_0,\ldots,V_t,\pi\right)$ with associated broadcast graph $D$, such that $V_{t-1}\neq V_t$, 
\begin{itemize}
  \item conditions \ref{def:boundary} and \ref{def:parent}--\ref{def:unique} of Definition \ref{def:broadcasttime} are satisfied,
  \item for each source $u\in S$ and each $j=1,\ldots,\min\{d,t-1\}$, there exists a node $v\in V_j\setminus V_{j-1}$ such that $\pi(v)=u$, and
  \item for each $k\in\{1,\ldots,t-2\}$, each node $u\in V_k\setminus V_{k-1}$, and each $j=k+1,\ldots,\min\{k+d-1,t-1\}$,
        there exists a node $v\in V_j\setminus V_{j-1}$ such that $\pi(v)=u$.
\end{itemize}
\noindent
Clearly, such a solution exists, and $\left(V_0,\ldots,V_t,\pi\right)$ is optimal if $\pi$ also satisfies condition \ref{def:edge} of Definition \ref{def:broadcasttime}.
We thus have $t\leq\tau(G,S)$.
It remains to prove that $2\sigma\sum_{k=1}^{t-1}f_k^{d-1}<n\leq 2\sigma\sum_{k=1}^tf_k^{d-1}$.

For $k=1,\ldots,t$, let $L_k=\left\{v\in V_k:\delta_{D_k}(v)=1\right\}$ denote the set of nodes with exactly one out- or in-neighbor in $D_k$,
and let $L_k=\emptyset$ for $k\leq 0$.
That is, for $i>1$, $L_k$ is the set of nodes that receive the signal in period $i$, whereas $L_1$ consists of all nodes informed in period 1, including the sources $S$.
Hence, $L_1,\ldots,L_{t-1}$ are disjoint sets (but $L_t$ may intersect $L_{t-1}$), and $V_k=L_1\cup\cdots\cup L_k$ for all $k=1,\ldots,t$.

Consider a period $k\in\{2,\ldots,t-1\}$.
The assumptions on $\left(V_0,\ldots,V_t,\pi\right)$ imply that $\pi$ is a bijection from $L_k$ to $L_{k-1}\cup\cdots\cup L_{k-d+1}$.
Thus, $\left|L_k\right|=\sum_{j=1}^{d-1}\left|L_{k-j}\right|$.
Since also $\left|L_1\right|=2\sigma=2\sigma f_1^{d-1}$ and $\left|L_j\right|=f_j^{d-1}=0$ for $j\leq 0$,
we get $\left|L_k\right|=2\sigma f_k^{d-1}$.
Further, $\left|L_t\right|\leq\sum_{j=1}^{d-1}\left|L_{t-j}\right|=2\sigma f_t^{d-1}$.
It follows that $2\sigma\sum_{k=1}^{t-1}f_k^{d-1}=\sum_{k=1}^{t-1}\left|L_k\right|=\left|V_{t-1}\right|<n=\left|V_t\right|\leq\sum_{k=1}^t\left|L_k\right|\leq 2\sigma\sum_{k=1}^tf_k^{d-1}$,
which completes the proof.
\end{proof}

\subsection{Continuous relaxations of integer programming models} \label{sec:lblprel}

For $t\in\mathbb{Z}_+$, define $\Omega(t)\subseteq[0,1]^{\overrightarrow{E}\times\{1,\ldots,t\}}$ as the set of feasible solutions to the continuous relaxation of
\eqref{mod:basic:dec:obj}--\eqref{mod:basic:dec:dim},
and let
\[
 \Omega^=(t) = \left\{x\in\Omega(t): \sum\limits_{u \in N(v)} \sum\limits_{k=1}^tx_{uv}^k = 1 ~~(v\in V\setminus S)\right\}.
\]
Let $t^{\ast}=\min\left\{t\in\mathbb{Z}_+: \Omega^=\neq\emptyset\right\}$ be the smallest value of $t$ for which the optimal objective function value in
the relaxation equals $n-\sigma$.
Existence of $t^{\ast}$ follows directly from $\Omega^=\left(\tau(G,S)\right)\neq\emptyset$.

% Consider a modification of model \eqref{mod:basic:dec:obj}--\eqref{mod:basic:dec:dim} where the coverage constraint
% $\sum\limits_{u \in N(v)} \sum\limits_{k=1}^tx_{uv}^k = 1$ is imposed on all non-source nodes $v\in V\setminus S$, and the objective is to minimize the communication time.
The continuous relaxation of \eqref{mod:basic:obj}--\eqref{mod:basic:dim} is feasible for sufficiently large $t$.
We denote its optimal objective function value by $\zeta(t)$.
% $\zeta(t)=\min\left\{\sum\limits_{k=1}^tz_k: x\in\Omega^=(t), z_k\geq\sum\limits_{u \in N(v)}x_{uv}^k ~~(k=1,\ldots,t, v\in V\setminus S)\right\}$.

\begin{proposition} \label{prop:lpweak}
For all $t\in\mathbb{Z}_+$ such that $\Omega^=(t)\neq\emptyset$, $\zeta(t)\leq t^{\ast}\leq\tau(G,S)$.
\end{proposition}
\begin{proof}
We first prove that $\zeta(t)$ is non-increasing with increasing $t$:
Let $\Gamma(t)$ denote the set of feasible solutions to the continuous relaxation of \eqref{mod:basic:obj}--\eqref{mod:basic:dim}, and assume $(x,z)\in\Gamma(t)$.
Define $\hat{x}\in[0,1]^{\overrightarrow{E}\times\{1,\ldots,t+1\}}$ such that
for all $(u,v)\in \overrightarrow{E}$, $\hat{x}_{uv}^k=x_{uv}^k$ ($k\leq t$) and $\hat{x}_{uv}^{t+1}=0$.
An analogous extension of $z$ to $\hat{z}\in[0,1]^{\{1,\ldots,t+1\}}$ yields $(\hat{x},\hat{z})\in\Gamma(t+1)$,
and $\sum_{k=1}^{t+1}\hat{z}_k=\sum_{k=1}^tz_k$ proves that $\zeta(t+1)\leq\zeta(t)$.

For $t\in\mathbb{Z}_+$ such that $\Omega^=(t)\neq\emptyset$, $t\geq t^{\ast}$ thus implies $\zeta(t)\leq\zeta(t^{\ast})$.
Since the only lower bounds on $z_k$ in \eqref{mod:basic:obj}--\eqref{mod:basic:dim} are $\max_{v\in V\setminus S}\sum\limits_{u \in N(v)}x_{uv}^k\leq 1$ and $z_{k-1}\leq 1$,
we get $\zeta(t)\leq\zeta(t^{\ast})\leq t^{\ast}$.
The proof is complete by observing that $t^{\ast}\leq\tau(G,S)$ follows from $\Omega^=\left(\tau(G,S)\right)\neq\emptyset$.
\end{proof}

\begin{remark} \label{rem:lpweak}
To compute a lower bound on $\tau(G,S)$, Proposition \ref{prop:lpweak} suggests to solve a sequence of instances of the continuous relaxation of problem \eqref{mod:basic:dec:obj}--\eqref{mod:basic:dec:dim},
and stop by the first value of $t$ for which the optimal objective function value is $n-\sigma$.
Such an approach yields a lower bound ($t^{\ast}$) on $\tau(G,S)$,
which is no weaker than the bound achieved by solving the continuous relaxation of \eqref{mod:basic:obj}--\eqref{mod:basic:dim}.
\end{remark}

\begin{remark} \label{rem:otheropt}
Remark \ref{rem:lpweak} applies to a reformulation of \eqref{mod:basic:obj}--\eqref{mod:basic:dim}, where a unique integer variable $y$ replaces $z_1,\ldots,z_t$,
and the objective is to minimize $y$ subject to the constraints
$y\geq\sum\limits_{k=1}^tk\sum\limits_{u \in N(v)}x_{uv}^k ~~(v\in V\setminus S)$, \eqref{mod:basic:singlein}--\eqref{mod:basic:tIncreases},
and \eqref{mod:basic:positiveCost}--\eqref{mod:basic:dim}.
\end{remark}

\subsection{Combinatorial relaxations} \label{sec:lbcombrel}

Lower bounds on the broadcast time $\tau(G,S)$ are obtained by omitting one or more of the conditions imposed in Definition \ref{def:broadcasttime}.
For the purpose of strongest possible bounds, the relaxations thus constructed can be supplied with conditions that are redundant in the problem definition.

Recall from Section \ref{sec:def} that $D=(V,A)$ denotes the broadcast forest corresponding to $\left(V_0,\ldots,V_t,\pi\right)$.
Conditions \ref{def:boundary}--\ref{def:unique} of Definition \ref{def:broadcasttime} imply that
\begin{enumerate}
\setcounter{enumi}{4}
  \item for all $v\in V$, $\delta_D^+(v)+\delta_D^-(v)\leq\delta_G(v)$. \label{def:degree}
\end{enumerate}

\noindent
A lower bound on $\tau(G,S)$ is then given by the solution to:
\begin{problem}[\textsc{Node Degree Relaxation}]\label{prob:degree}
Find the smallest integer $t\geq 0$ for which there exist
a sequence $V_0\subseteq\dots\subseteq V_t$ of node sets and a function $\pi:V\setminus S\to V$,
satisfying conditions \ref{def:boundary} and \ref{def:parent}--\ref{def:degree}.
\end{problem}

Observe that the bound given in Proposition \ref{prop:lbfib} is obtained by exploiting the lower-bounding capabilities of the \textsc{Node Degree Relaxation}.
By considering the degree of all nodes $v\in V$, rather than just the maximum degree, stronger bounds may be achieved in instances where $G$ is not regular
($\min_{v\in V}\delta_G(v)<\max_{v\in V}\delta_G(v)$).

% In order to develop an exact algorithm for the \textsc{Node Degree Relaxation}, we derive some optimality conditions.
Denote the source nodes $S=\left\{v_1,\dots,v_{\sigma}\right\}$ and the non-source nodes $V\setminus S=\left\{v_{\sigma+1},\ldots,v_n\right\}$, where $\delta_G(v_{\sigma+1})\geq\delta_G(v_{\sigma+2})\geq\dots\geq\delta_G(v_n)$,
and let $d_i=\delta_G(v_i)$ ($i=1,\ldots,n$).
Thus, $\left\{d_1,\ldots,d_n\right\}$ resembles the conventional definition of a non-increasing degree sequence of $G$,
with the difference that only the subsequence consisting of the final $n-\sigma$ degrees is required to be non-increasing.

For a given $t\in\mathbb{Z}_+$, consider the problem of finding $\left(V_0,\ldots,V_t,\pi\right)$ such that $V_0=S$,
conditions \ref{def:parent}--\ref{def:degree} are satisfied, and $\left|V_t\right|$ is maximized.
The smallest value of $t$ for which the maximum equals $n$ is obviously the solution to Problem~\ref{prob:degree}.

The algorithm for Problem~\ref{prob:degree}, to follow later in the section, utilizes that the maximum value of $\left|V_t\right|$
is achieved by transmitting the signal to nodes in non-increasing order of their degrees.
Observe that, contrary to the case of Problem~\ref{prob:min}, transmissions to non-neighbors are allowed in the relaxed problem.
Any instance of Problem~\ref{prob:degree} thus has an optimal solution where, for $k=1,\ldots,t-1$,
$u\in V_k\setminus V_{k-1}$ and $v\in V_{k+1}\setminus V_k$ implies $\delta_G(u)\geq\delta_G(v)$.

A rigorous proof of this follows next.

\begin{lemma}
\label{lemma:degchild}
For a positive integer $t$ and node sets $V_0\subseteq V_1\subseteq\cdots\subseteq V_t$, where $V_0=S$,
there exists a $\pi:V_t\setminus S\mapsto V_t$ such that $\left(V_0,\ldots,V_t,\pi\right)$ satisfies 
conditions \ref{def:parent}--\ref{def:degree} if and only if, for all $k=1,\ldots,t$,
\begin{equation} \label{eq:lemiff}
\left|V_k\setminus S\right| \leq \sum_{v_i\in S}\min\{d_i, k\}+\sum_{\ell=1}^{k-1}\sum_{v_i\in V_{\ell}\setminus V_{\ell-1}}\min\{d_i-1, k-\ell\}.
\end{equation}
\end{lemma}
\begin{proof}
Consider the broadcast forest $D_k$ corresponding to some feasible $\left(V_0,\ldots,V_k,\pi\right)$.
Condition \ref{def:unique} implies that every node $v_i\in V_{\ell}\setminus V_{\ell-1}$ ($1\leq\ell<k$) has at most one child node in each of the node sets
$V_{\ell+1},\ldots,V_k$, and thus no more than $k-\ell$ child nodes in $D_k$.
Condition \ref{def:degree} implies that $v_i$ has at most $d_i-1$ child nodes in $D_k$.
The corresponding bounds for a source $v_i\in V_0$ are $k$ and $d_i$, respectively.
Conversely, the conditions \ref{def:unique}--\ref{def:degree} are satisfied for some $\pi$ if both bounds are respected at all nodes.
The proof is complete by observing that condition \ref{def:parent} is satisfied if and only if the number $\left|V_k\setminus S\right|$ of non-root nodes in $D_k$
is no larger than the total child node capacity $\sum_{v_i\in S}\min\{d_i, k\}+\sum_{\ell=1}^{k-1}\sum_{v_i\in V_{\ell}\setminus V_{\ell-1}}\min\{d_i-1, k-\ell\}$ in $D_k$.
\end{proof}

\begin{lemma}
\label{lemma:degorder}
The maximum value of $\left|V_t\right|$ over all $\left(V_0,\ldots,V_t,\pi\right)$ satisfying $V_0=S$ and
conditions \ref{def:parent}--\ref{def:degree}, is attained by some
$\left(V_0,\ldots,V_t,\pi\right)$ where \\
$\min\left\{i: v_i\in V_{k}\setminus V_{k-1}\right\}>\max\left\{i: v_i\in V_{k-1}\right\}$ ($k=1,\ldots,t$).
\end{lemma}
\begin{proof}

Consider an arbitrary optimal solution $\left(V_0,\ldots,V_t,\pi\right)$,
and assume that $v_i\in V_p\setminus V_{p-1}$, $v_j\in V_{q}\setminus V_{q-1}$, $i<j$, and $1\leq q<p\leq t$.
We prove that the solution 
obtained by swapping nodes $v_i$ and $v_j$ is also optimal.
Let $V_k'=V_k$ for $k=0,\ldots,q-1, p, p+1,\ldots,t$, and $V_k'=\left(V_k\setminus\{v_j\}\right)\cup\{v_i\}$ for $k=q,\ldots,p-1$.
Because $\left|V_t'\right|=\left|V_t\right|$, we only need to show that $\left(V_0',\ldots,V_t',\pi'\right)$ is feasible for some $\pi'$.

By feasibility of $\left(V_0,\ldots,V_t,\pi\right)$, the only-if part of Lemma \ref{lemma:degchild} implies that \eqref{eq:lemiff} holds for $k=1,\ldots,t$.
The if part of the same lemma shows that if \eqref{eq:lemiff} applies to $\left(V_0',\ldots,V_t'\right)$, too, it is also feasible.

Let $\alpha_k$ and $\alpha_k'$ denote the right hand side of \eqref{eq:lemiff} corresponding to
$\left(V_0,\ldots,V_t\right)$ and $\left(V_0',\ldots,V_t'\right)$, respectively.
Then $\alpha_k'=\alpha_k$ for $k=1,\ldots,q-1$.
For $k=q+1,\ldots,p$, we have $\alpha_k'=\alpha_k+\min\{d_i-1, k-q\}-\min\{d_j-1, k-q\}\geq \alpha_k$, since $d_i\geq d_j$.
For $k=p+1,\ldots,t$, we have $\alpha_k'=\alpha_k+\min\{d_i-1, k-q\}+\min\{d_j-1, k-p\}-\min\{d_j-1, k-q\}-\min\{d_i-1, k-p\}$.
It thus remains to show that $K_1+L_1\geq K_2+L_2$, where
$$
K_1=\min\{d_i-1,k-q\}, ~~~ L_1=\min\{d_j-1,k-p\},
$$
and
$$
K_2=\min\{d_i-1,k-p\}, ~~~ L_2=\min\{d_j-1,k-q\}.
$$
We now investigate what values the difference 
\begin{equation}
\label{eq:degbounds}
(K_1+L_1)-(K_2+L_2)
\end{equation}
can attain.
Observe that $K_1\geq L_2$ and $K_2\geq L_1$. 
If $K_2>L_1$, there are two distinct cases:
If $K_2=d_i-1=K_1$, then also $L_2=d_j-1=L_1$.
Otherwise, $K_2=k-p\leq K_1$, implying $L_1 =d_j-1$, and since $d_j-1\leq k-p\leq k-q$, we have $L_2=d_j-1=L_1$.
We conclude that \eqref{eq:degbounds} always takes a non-negative value, which completes the proof.
\end{proof}

Alg. \ref{alg:dreg} operates with the set $I(t)$ that consists of indices of nodes informed within the first $t$ time steps.
The function $a_i(t)$ determines the number of nodes informed by node $v_i$ within $t$ time steps, and finally,
the set $F(t)$ consists of nodes that in period $t$ can inform other nodes.
That is, $v\in F(t)$ if $v$ informed less than $\delta_G(v)$ other nodes, and $v$ itself is also informed.

\begin{algorithm}
\KwData{$I(0)=\{1,\dots,\sigma\}, a_1(0)=\dots = a_{n}(0)=0.$}
\For{$t=1,2,\dots$} {
	$F(t)=\{i\in I(t-1):a_i(t-1)<d_i\},$\\
	$I(t)=\{1,\dots,|I(t-1)|+|F(t)|\},$\\
	$a_i(t)=a_i(t-1)+
	\begin{cases}
		1, i\in F(t)\cup \left(I(t)\setminus I(t-1)\right) \\
		0, \text{ otherwise. }\\
	\end{cases}$\\
	\If{$|I(t)|=n$} {\Return $t$}
		
}
\caption{Lower bound exploiting distribution of degrees}
\label{alg:dreg}
\end{algorithm}


\begin{proposition}
Alg.~\ref{alg:dreg} returns a lower bound on $\tau(G,S)$.
\label{cor:deg}
\end{proposition}
\begin{proof}
		Follows from Lemma \ref{lemma:degorder} and the subsequent discussion. 
\end{proof}

